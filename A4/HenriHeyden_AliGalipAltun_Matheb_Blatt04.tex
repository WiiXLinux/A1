\documentclass[12pt, a4paper]{article}

\usepackage[ngerman]{babel} 
\usepackage[T1]{fontenc}
\usepackage{amsfonts} 
\usepackage{setspace}
\usepackage{amsmath}
\usepackage{amssymb}

\newcommand*{\qed}{\null\nobreak\hfill\ensuremath{\square}}
\newcommand*{\puffer}{\text{ }\text{ }\text{ }\text{ }}

\pagestyle{plain}
\allowdisplaybreaks

\title{Mathematik für die Informatik B - Hausaufgabenserie 2}
\author{Henri Heyden, Ali Galip Altun \\ \small stu240825, stu242631}
\date{}


\begin{document}
\maketitle


\doublespacing

\section*{Aufgabe 1}
\textbf{Beh.:} Die Aussage beschrieben in der Aufgabenstellung gilt nicht, das heißt es gilt: \(\exists (x_n)_n,(y_n)_n \in \mathcal{S}(\mathbb R_{\ne 0}), \lim_{n}x_n = \lim_{n}y_n = +\infty: \lim_{n}\frac{x_n}{y_n} \not\in \overline{\mathbb{R}}\) \\
\textbf{Vor.:} \(n \in \mathbb{N}\). \\
\textbf{Bew.:} Sei \(a_n := n + n^2 + n^2\cdot(-1)^n\) und \(b_n := n^2\).\\
Dann gilt: \(\frac{a_n}{b_n} = \frac{n + n^2 + n^2\cdot(-1)^n}{n^2} = \frac{1}{n}+1+(-1)^n\). \\
Wir werden nun den Limes dieser Folge versuchen zu berechnen. Betrachte folgende Umformung:
\begin{flalign*}
    & \lim_{n} \frac{a_n}{b_n} = \lim_{n} \left(\frac{1}{n}+1+(-1)^n\right) & \text{| Kombinationssätze} \\
    & \puffer \puffer \text{ } = \lim_{n} \frac{1}{n} + \lim_{n} 1 + \lim_{n} (-1)^n & \text{| Ausrechnen} \\
    & \puffer \puffer \text{ } = 0 + 1 + \lim_{n} (-1)^n & \text{| } \lim_{n} (-1)^n \not \in \overline{\mathbb{R}} \\
    & \puffer \puffer \text{ } \not\in \overline{\mathbb{R}}.
\end{flalign*}
Damit gilt was zu zeigen war, also gilt die Aussage aus der Aufgabenstellung nicht. \qed
\section*{Aufgabe 2}
\textbf{Beh.:} Die Aussage beschrieben in der Aufgabenstellung gilt nicht, das heißt es gilt: \(\exists (x_k)_{k\ge1}\in \mathcal{S}(\mathbb R), \lim_{n}\left(\frac{1}{n} \sum_{k=1}^{n} x_k\right)\in \mathbb R: \lim_k x_k \not\in \mathbb R\). \\
\textbf{Vor.:} \(n,k \in \mathbb{N}_1\). \\
\textbf{Bew.:} Sei \(x_k := (-1)^k\). Dann gilt: \(\left(\frac{1}{n} \sum_{k=1}^{n} x_k\right)_n = \left(\frac{1}{n} \sum_{k=1}^{n} (-1)^k)\right)_n\), wir nennen diese Mittelwertsfolge \(m_n\). \\
Wir werden zeigen, dass \(\lim_n m_n = 0 \in \mathbb R\) gilt.\\
Wir werden das \(\epsilon\)-Kriterium anwenden, das heißt wir werden zeigen, dass folgendes gilt: \(\forall \epsilon > 0: \exists n_0: \forall n \ge n_0: |m_n - 0| < \epsilon\). Setze \(n_0 := \left\lceil \frac{1}{n} \right\rceil + 1\). Dann gilt \(n_0 \in \mathbb N_1\).
Sei \(n \ge n_0\). Dann gilt folgende Umformung:
\begin{flalign*}
    & |m_n - 0| = |m_n| = \left|\frac{1}{n} \sum_{k=1}^{n} (-1)^k\right| & \text{| } \frac{1}{n} > 0 \\
    & = \frac{1}{n} \cdot \left|\sum_{k=1}^{n} (-1)^k\right| & \text{| Annahme } n \ge n_0 \text{ und } \frac{1}{n_0} < n_0 \\
    & \le \frac{1}{n_0} \cdot \left|\sum_{k=1}^{n_0} (-1)^k\right| & \text{| setze in \(n_0\) ein} \\
    & = \frac{1}{\left\lceil \frac{1}{n} \right\rceil + 1} \cdot \left|\sum_{k=1}^{n_0} (-1)^k\right| & \text{| Beobachte } \max\left\{\left|\sum_{k=1}^{n_0} (-1)^k\right|\right\} \\
    & \le \frac{1}{\left\lceil \frac{1}{n} \right\rceil + 1} \cdot 1 = \frac{1}{\left\lceil \frac{1}{k} \right\rceil + 1} &  = \max\{|-1|, |0|\} = 1 \\
    & < \frac{1}{\frac{1}{\epsilon}} = \epsilon.
\end{flalign*}
Somit haben wir gezeigt, dass \(\lim_n m_n = 0\) gilt und da \(0 \in \mathbb R\) gilt, konvergiert \((m_n)_n\).
Beobachte aber, dass \((x_k)_k\) nicht konvergiert nach Vorlesung.\\
Damit haben wir gezeigt, was zu zeigen war und die Aussage aus der Aufgabenstellung gilt nicht. \qed
\end{document}