\documentclass[12pt, a4paper]{article}

\usepackage[ngerman]{babel} 
\usepackage[T1]{fontenc}
\usepackage{amsfonts} 
\usepackage{setspace}
\usepackage{amsmath}
\usepackage{amssymb}
\usepackage{titling}


\newcommand*{\qed}{\null\nobreak\hfill\ensuremath{\square}}
\newcommand*{\puffer}{\text{ }\text{ }\text{ }\text{ }}
\newcommand*{\gedanke}{\textbf{-- }}


\pagestyle{plain}
\allowdisplaybreaks

\setlength{\droptitle}{-20em}
\setlength{\jot}{12pt}

\title{Mathematik für die Informatik B - Hausaufgabenserie 6}
\author{Henri Heyden, Ali Galip Altun \\ \small stu240825, stu242631}
\date{}


\begin{document}
\maketitle

\doublespacing
\section*{Aufgabe 1}
\textbf{Beh.:} Die Reihe über \(\left(\frac{c^n}{n!}\right)_n\) konvergiert absolut. \\
\textbf{Vor.:} \(n \in \mathbb N, c > 0\) \\
\textbf{Bew.:} Nach dem Quotientenkriterium (Limes-Version) ist zu zeigen, dass \(\lim_{n} \left| \frac{\left(\frac{c^n}{n!}\right)}{\left(\frac{c^{n-1}}{(n-1)!}\right)} \right| < 1\) gilt. \\
Betrachte hierfür folgende Vereinfachung:
\begin{flalign*}
    & \puffer \lim_{n} \left| \frac{\left(\frac{c^n}{n!}\right)}{\left(\frac{c^{n-1}}{(n-1)!}\right)} \right| & \text{| Def. Potenz und Def. Fakultät} \\
    & = \lim_{n} \left| \frac{\left(\frac{c \cdot c^{n-1}}{n \cdot (n-1)!}\right)}{\left(\frac{c^{n-1}}{(n-1)!}\right)} \right| & \text{| Bruchrechnung} \\
    & = \lim_{n} \left| \frac{c}{n} \cdot \frac{\left(\frac{c^{n-1}}{(n-1)!}\right)}{\left(\frac{c^{n-1}}{(n-1)!}\right)} \right| & \text{| kürzen} \\
    & = \lim_{n} \left| \frac{c}{n} \right| & \text{| n > 0, c > 0} \\
    & = \lim_{n} \frac{c}{n} & \text{| Kombinationssätze, \(\lim_{n} c = c\)} \\
    & = c \cdot \lim_{n} \frac{1}{n} & \text{| Auswerten} \\
    & = c \cdot 0 & \\
    & = 0 & \\
    & < 1
\end{flalign*}
Damit gilt, was zu zeigen war, also konvergiert die Reihe über \(\left(\frac{c^n}{n!}\right)_n\) absolut. \qed
\end{document}