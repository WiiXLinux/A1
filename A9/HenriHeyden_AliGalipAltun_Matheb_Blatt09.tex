\documentclass[12pt, a4paper]{article}

\usepackage[ngerman]{babel} 
\usepackage[T1]{fontenc}
\usepackage{amsfonts} 
\usepackage{setspace}
\usepackage{amsmath}
\usepackage{amssymb}
\usepackage{titling}


\newcommand*{\qed}{\null\nobreak\hfill\ensuremath{\square}}
\newcommand*{\puffer}{\text{ }\text{ }\text{ }\text{ }}
\newcommand*{\gedanke}{\textbf{-- }}
\newcommand*{\GL}{\ \overset{>}{{}_\text{\scriptsize \(<\)}}\ } % Very ugly solution
\newcommand*{\LG}{\ \overset{<}{{}_\text{\scriptsize \(>\)}}\ } % but it just works

\pagestyle{plain}
\allowdisplaybreaks

\setlength{\droptitle}{-11em}
\setlength{\jot}{12pt}

\title{Mathematik für die Informatik B - Hausaufgabenserie 9}
\author{Henri Heyden, Ali Galip Altun \\ \small stu240825, stu242631}
\date{}


\begin{document}
\maketitle

\doublespacing
\section*{Aufgabe 1}
\textbf{Vor.:} \(n \in \mathbb{N}_1\), \(f: [0, +\infty[\ \rightarrow\ \mathbb{R}, x \mapsto x^ne^{-x}\) \\
\(I_1 := [0, n]\), \(I_2 := [n, +\infty[\) \\
\textbf{Beh.:} \(f\) ist streng monoton steigend in \(I_1\), streng monoton fallend in \(I_2\) und es gilt: \(\text{LMAX}(f) = {n}\) \\
\textbf{Bew.:} \(f\) ist über Kombinationssätze der Differenzierbarkeit differenzierbar in \(dom(f)\) mit der Ableitung:
\begin{flalign*}
    & f'(x) & \text{| Quotientenregel, Bruchrechnung} \\
    & = nx^{n-1} \cdot e^{-x} + x^n \cdot \frac{- e^x}{(e^x)^2} & \text{| Bruchrechnung} \\
    & = nx^{n-1} \cdot e^{-x} - x^n \cdot \frac{1}{e^x} & \text{| Potenzgesetze} \\
    & = nx^{n-1} \cdot e^{-x} - x^n \cdot e^{-x} & \text{| Ausklammern} \\
    & = \left( nx^{n-1} - x^n \right) \cdot e^{-x}
\end{flalign*} für jedes \(x \in dom(f)\). \\
Nach der Vorlesung ist bekannt, dass \(e^x > 0\) gilt, für \(x \in dom(f)\), somit ist für die Suche nach Nullstellen oder für das Vorzeichenverhalten von \(f'\) an einer Stelle \(x\) der Term "\(e^{-x}\)", der in \(f'(x)\) liegt, zu vernachlässigen, da somit auch \(e^{-x} = (e^x)^{-1} > 0\) durch die Regeln in angeordneten Körpern gilt. \\
Wir werden nun die Ungleichungen \(f'(x) > 0\) und \(f'(x) < 0\) betrachten. Um dies etwas zu verkürzen, schreiben wir "\(f'(x) \GL 0\)" \ und betrachten damit beide Fälle, also beide Ungleichungen. Um die Bedeutung der Reihenfolge festzulegen, ist \(f'(x) \GL 0\) äquivalent zu \(0 \LG f'(x)\), genau wie der Ausdruck "\(a = 1 \pm -1\)"\ äquivalent zu "\(a = 1 \mp 1\)"\ ist. \\
Damit betrachten wir die Ungleichungen, \(f'(x) \GL 0\):
\begin{flalign*}
    & \ \ f'(x) \GL 0 & \text{| Schreibweise, Einsetzen, \(e^x > 0\)} \\
    & \Longleftrightarrow 0 \LG n \cdot x^{n-1} - x^n & \text{| Potenzgesetze} \\
    & \Longleftrightarrow 0 \LG n \cdot x^{n-1} - x \cdot x^{n-1} & \text{| Ausklammern} \\
    & \Longleftrightarrow 0 \LG (n-x) \cdot x^{n-1} & \text{| Wir betrachten nicht \(x = 0\) (dazu später noch was), \dots} \\
    & & \text{\dots deswegen können wir\ \(x^{n-1} = 0\) ignorieren, \dots} \\
    & & \text{\dots und im Fall \(n = 1\) gilt \(x^{n-1} = 1\)} \\
    & \Longrightarrow 0 \LG n - x & \text{| Schreibweise} \\
    & \Longleftrightarrow n - x \GL 0 & \text{| \(+x\), \(\cdot -1\)} \\
    & \Longleftrightarrow x \LG n
\end{flalign*}
Somit gilt also \(f'(x) > 0\), wenn \(x < n\) gilt und \(f'(x) < 0\), wenn \(x < n\) gilt. \\
Insbesondere gilt \(f'(x) = 0\), wenn \(x = n\) gilt, was aus einer ähnlichen Umformung folgt (Beachte hier immer noch, dass wir \(x = 0\) nicht betrachten, denn es gilt \(\forall x \in \left]-\infty, 0\right[: x \not\in dom(f)\) und somit kann 0 nicht innerer Punkt von \(dom(f)\) sein und somit kann keine Extremstelle von \(f\) vorliegen an 0, da keine Umgebung um 0 existiert). \\
Aufgrund des Monotoniekriteriums sind die ersten beiden Aussagen, die zu zeigen waren gezeigt, und da \(x = n\) die einzige Nullstelle von \(f'\) ist, die wir betrachten können, gilt nun auch die dritte Aussage, die zu zeigen war nach Einsatztechnik. \\
Somit ist alles gezeigt, was zu zeigen war \qed
\end{document}