\documentclass[12pt, a4paper]{article}

\usepackage[ngerman]{babel} 
\usepackage[T1]{fontenc}
\usepackage{amsfonts} 
\usepackage{setspace}
\usepackage{amsmath}
\usepackage{amssymb}
\usepackage{titling}


\newcommand*{\qed}{\null\nobreak\hfill\ensuremath{\square}}
\newcommand*{\puffer}{\text{ }\text{ }\text{ }\text{ }}
\newcommand*{\gedanke}{\textbf{-- }}
\newcommand*{\GL}{\ \overset{>}{{}_\text{\scriptsize \(<\)}}\ } % Very ugly solution
\newcommand*{\LG}{\ \overset{<}{{}_\text{\scriptsize \(>\)}}\ } % but it just works

\pagestyle{plain}
\allowdisplaybreaks

\setlength{\droptitle}{-11em}
\setlength{\jot}{12pt}

\title{Mathematik für die Informatik B - Hausaufgabenserie 9}
\author{Henri Heyden, Ali Galip Altun \\ \small stu240825, stu242631}
\date{}


\begin{document}
\maketitle

\doublespacing
\section*{Aufgabe 1}
\textbf{Vor.:} \(n \in \mathbb{N}_1\), \(f: [0, +\infty[\ \rightarrow\ \mathbb{R}, x \mapsto x^ne^{-x}\) \\
\(I_1 := [0, n]\), \(I_2 := [n, +\infty[\) \\
\textbf{Beh.:} \(f\) ist streng monoton steigend in \(I_1\), streng monoton fallend in \(I_2\) und es gilt: \(\text{LMAX}(f) = {n}\) \\
\textbf{Bew.:} \(f\) ist über Kombinationssätze der Differenzierbarkeit differenzierbar in \(dom(f)\) mit der Ableitung:
\begin{flalign*}
    & f'(x) & \text{| Quotientenregel, Bruchrechnung} \\
    & = nx^{n-1} \cdot e^{-x} + x^n \cdot \frac{- e^x}{(e^x)^2} & \text{| Bruchrechnung} \\
    & = nx^{n-1} \cdot e^{-x} - x^n \cdot \frac{1}{e^x} & \text{| Potenzgesetze} \\
    & = nx^{n-1} \cdot e^{-x} - x^n \cdot e^{-x} & \text{| Ausklammern} \\
    & = \left( nx^{n-1} - x^n \right) \cdot e^{-x}
\end{flalign*} für jedes \(x \in dom(f)\). \\
Nach der Vorlesung ist bekannt, dass \(e^x > 0\) gilt, für \(x \in dom(f)\), somit ist für die Suche nach Nullstellen oder für das Vorzeichenverhalten von \(f'\) an einer Stelle \(x\) der Term "\(e^{-x}\)", der in \(f'(x)\) liegt, zu vernachlässigen, da somit auch \(e^{-x} = (e^x)^{-1} > 0\) durch die Regeln in angeordneten Körpern gilt. \\
Wir werden nun die Ungleichungen \(f'(x) > 0\) und \(f'(x) < 0\) betrachten. Um dies etwas zu verkürzen, schreiben wir "\(f'(x) \GL 0\)" \ und betrachten damit beide Fälle, also beide Ungleichungen. Um die Bedeutung der Reihenfolge festzulegen, ist \(f'(x) \GL 0\) äquivalent zu \(0 \LG f'(x)\), genau wie der Ausdruck "\(a = 1 \pm -1\)"\ äquivalent zu "\(a = 1 \mp 1\)"\ ist. \\
Damit betrachten wir die Ungleichungen, \(f'(x) \GL 0\):
\begin{flalign*}
    & \ \ f'(x) \GL 0 & \text{| Schreibweise, Einsetzen, \(e^x > 0\)} \\
    & \Longleftrightarrow 0 \LG n \cdot x^{n-1} - x^n & \text{| Potenzgesetze} \\
    & \Longleftrightarrow 0 \LG n \cdot x^{n-1} - x \cdot x^{n-1} & \text{| Ausklammern} \\
    & \Longleftrightarrow 0 \LG (n-x) \cdot x^{n-1} & \text{| Wir betrachten nicht \(x = 0\) (dazu später noch was), \dots} \\
    & & \text{\dots deswegen können wir\ \(x^{n-1} = 0\) ignorieren, \dots} \\
    & & \text{\dots und im Fall \(n = 1\) gilt \(x^{n-1} = 1\)} \\
    & \Longrightarrow 0 \LG n - x & \text{| Schreibweise} \\
    & \Longleftrightarrow n - x \GL 0 & \text{| \(+x\), \(\cdot -1\)} \\
    & \Longleftrightarrow x \LG n
\end{flalign*}
Somit gilt also \(f'(x) > 0\), wenn \(x < n\) gilt und \(f'(x) < 0\), wenn \(x < n\) gilt. \\
Insbesondere gilt \(f'(x) = 0\), wenn \(x = n\) gilt, was aus einer ähnlichen Umformung folgt (Beachte hier immer noch, dass wir \(x = 0\) nicht betrachten, denn es gilt \(\forall x \in \left]-\infty, 0\right[: x \not\in dom(f)\) und somit kann 0 nicht innerer Punkt von \(dom(f)\) sein und somit kann keine Extremstelle von \(f\) vorliegen an 0, da keine Umgebung um 0 existiert). \\
Aufgrund des Monotoniekriteriums sind die ersten beiden Aussagen, die zu zeigen waren gezeigt, und da \(x = n\) die einzige Nullstelle von \(f'\) ist, die wir betrachten können, gilt nun auch die dritte Aussage, die zu zeigen war nach Einsatztechnik. \\
Somit ist alles gezeigt, was zu zeigen war \qed
\section*{Aufgabe 2}
\textbf{Vor.:} Sei \(\Omega \subseteq \mathbb C\), \(z \in \Omega\) ein HP,\\
\(f,g \in \mathbb{C}^\Omega\), sodass: \(\lim_{\zeta \rightarrow z} |f(\zeta)| = +\infty\ \wedge\ \lim_{\zeta \rightarrow z}(fg)(\zeta) \in \mathbb C\) gilt. \\
Seien außerdem \((a_n)_n := (f(\zeta_n))_n\) und \((b_n)_n := (g(\zeta_n))_n\) die zugehörigen Funktionswertfolgen. \\
\textbf{Beh.:} \(\lim_{\zeta \rightarrow z} g(\zeta) = 0\) (Nach Voraussetzung äquivalent zu: \(\lim_{n} b_n = 0\)) \\
\textbf{Bew.:} Nach der Voraussetzung können wir die angenommenen Aussagen umschreiben in: \(\lim_{n} |a_n| = +\infty\) und \(\lim_{n} (a_n \cdot b_n) \in \mathbb{C}\). \\
Nach der Mindestindex-Charakterisierung für komplexe Folgen ist die zweite Aussage dann äquivalent zu: \[\forall \epsilon > 0: \exists n_0: \forall n \ge n_0: |a_n \cdot b_n - p| < \epsilon\]
wobei \(p \in \mathbb C\) der Limes ist, zu dem die kombinierte Folge konvergiert. \\
Betrachte nun folgende Umformung der Ungleichung \(|a_n \cdot b_n - p| < \epsilon\) für die gegebenen Voraussetzungen:
\begin{flalign*}
    & \epsilon & \\
    & > |a_n \cdot b_n - |p|| & \text{| S 4.8 (15): Dreiecksungleichung nach unten} \\
    & \ge ||a_n \cdot b_n| - |p|| & \text{| S 4.8 (11)} \\
    & = ||a_n| \cdot |b_n| - |p||
\end{flalign*}
Wir wissen, dass ab einem Index die Folge \((|a_n|)_n\) monoton steigend ist.\\
Des Weiteren wissen wir, dass \(|p| \in \mathbb R\) gilt. \\
Somit muss \((|b_n|)_n\) monoton fallend sein, was durch den Betrag nur funktioniert, wenn \((b_n)_n\) gegen null konvergiert.\\
Hört \((b_n)_n\) auf sich an 0 anzunähren, kann \((|b_n|)_n\) nicht gegen null konvergieren womit es einen Index geben wird, ab dem die kombinierte Folge\\
(also \((a_n \cdot b_n)_n\)) gegen \(+\infty\) konvergiert, da sie ab diesem Index wieder anfinge monoton zu steigen. \\
Damit ist \(\lim_n b_n = 0\) gezeigt, \gedanke was zu zeigen war. \qed
\section*{Aufgabe 3}
\end{document}