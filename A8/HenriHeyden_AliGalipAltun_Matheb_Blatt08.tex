\documentclass[12pt, a4paper]{article}

\usepackage[ngerman]{babel} 
\usepackage[T1]{fontenc}
\usepackage{amsfonts} 
\usepackage{setspace}
\usepackage{amsmath}
\usepackage{amssymb}
\usepackage{titling}


\newcommand*{\qed}{\null\nobreak\hfill\ensuremath{\square}}
\newcommand*{\puffer}{\text{ }\text{ }\text{ }\text{ }}
\newcommand*{\gedanke}{\textbf{-- }}


\pagestyle{plain}
\allowdisplaybreaks

\setlength{\droptitle}{-11em}
%\setlength{\jot}{12pt}

\title{Mathematik für die Informatik B - Hausaufgabenserie 8}
\author{Henri Heyden, Ali Galip Altun \\ \small stu240825, stu242631}
\date{}


\begin{document}
\maketitle

\doublespacing

\section*{Aufgabe 1}
\textbf{Vor.:} \(f: \mathbb R \setminus \{\pm \sqrt{2}\} \rightarrow \mathbb R, x \mapsto \frac{3x^2+x-8}{x^2-2}\) \\
\textbf{Beh.:} \(\text{LMAX}(f) = \{2 + \sqrt{2}\}, \text{ LMIN}(f) = \{2 - \sqrt{2}\}\), \\
\(f\) ist streng monoton fallend in \(\left]-\infty, -\sqrt{2}\right[\), \\
streng monoton fallend in \(] - \sqrt{2}, 2 - \sqrt{2}[\),\\
streng monoton steigend in \(]2 - \sqrt{2}, \sqrt{2}[\), \\
streng monoton steigend in \(]\sqrt{2}, 2+\sqrt{2}[\) und \\
streng monoton fallend in \(]2+\sqrt{2}, +\infty[\).\\
\textbf{Bew.:} Nach Vorlesung ist \(f\) differenzierbar in ihrer Domain mit der Ableitung (Quotientenregel): \(f'(x) = \frac{(6x + 1) \cdot (x^2-2) - (3x^2+x-8)\cdot 2x}{(x^2-2)^2}\). Wenn man dies kürzt indem man alle Klammern auflöst und vereinfacht, ergibt sich: \(f'(x) = \frac{-x^2 + 4x - 2}{(x^2 - 2)^2}\). Nach Satz 3.48 müssen Nullstellen von \(f'(x)\) existieren, sodass \(f\) Extremstellen haben kann, also suchen wir Lösungen für \(f'(x) = 0\).\\
Betrachte folgende Umformung: \pagebreak
\begin{flalign*}
    & 0 = f'(x) & \text{} \\
    & 0 = \frac{-x^2 + 4x - 2}{(x^2 - 2)^2} & \text{| \(\cdot (x^2 - 2)^2\), \(x \ne \pm \sqrt{2}\)} \\
    & \Longleftrightarrow 0 = {-x^2 + 4x - 2} & \text{| Lösung von einer quadratischen Gleichung} \\
    & \Longleftrightarrow x_{1,2} = \frac{-4}{-2} \pm \frac{\sqrt{16-8}}{-2} & \text{| Auswerten} \\
    & x_{1,2} = 2 \mp \sqrt{2} & 
\end{flalign*}
Nun wissen wir, dass die Stellen \(x = 2 \pm \sqrt{2}\) möglicherweise Extremstellen sein können. Gleichzeitig sind dies die einzigen Stellen, die Extremstellen sein können, aufgrund vom Satz 3.48. \\
Da \(f\) vollständig differenzierbar ist, und somit auch vollständig stetig ist in ihrer Domain, betrachten wir den Differenzialwert von \(f'\) an den Stellen \(f(2 \pm \sqrt{2})\) um herauszufinden, ob \(f'\) einen Vorzeichenwechsel um diese Stelle hat, und wenn, was für ein VZW vorliegt. \\
\(f'\) ist vollständig differenzierbar in jedem Punkt außer \(\sqrt{2}\) (nach Eigenschaften von rationalen Funktionen) mit der Ableitung \[f''(x) = \frac{(-2x+4) \cdot (x^2-2)^2 - (-x^2 + 4x -2) \cdot (4x^3-8x)}{(x^2-2)^4}\]
Dies folgt aus der Quotientenregel und den Potenzgesetzen. \\
Beobachte: \(f''(2 + \sqrt{2}) < 0\) und \(f''(2 - \sqrt{2}) > 0\). \\
Somit hat \(f'\) einen VZW von \(+ \rightarrow -\) und von \(- \rightarrow +\) in den Stellen \(2 + \sqrt{2}\) bzw. \(2 - \sqrt{2}\). \\
Hieraus folgt, dass\dots \pagebreak \\
\textbf{1.:} \(f\) ein lokales Maximum in \(2 + \sqrt{2}\) besitzt, da die Umgebung \(]\sqrt{2}, +\infty[\) nur kleinere Werte enthält, was daraus folgt, dass \(f\) streng monoton steigend ist im Intervall \(]\sqrt{2}, 2+\sqrt{2}[\) und streng monoton fallend ist im Intervall \(]2+\sqrt{2}, +\infty[\). \\
Dies beides folgt aus den Eigenschaften von \(f'\), die wir gerade bestimmt hatten. \\
\textbf{2.:} \(f\) ein lokales Minimum in \(2 - \sqrt{2}\) besitzt, da die Umgebung \(]- \sqrt{2}, + \sqrt{2}\) nur größere Werte enthält, was daraus folgt, dass \(f\) streng monoton fallend ist im Intervall \(] - \sqrt{2}, 2 - \sqrt{2}[\) und streng monoton steigend ist im Intervall \(]2 - \sqrt{2}, \sqrt{2}[\). \\
Dies beides folgt aus den Eigenschaften von \(f'\), die wir gerade bestimmt hatten. \\
Nun ist das Extremverhalten von \(f\) geklärt.\\
Des Weiteren sind wir nun aufgeklärt über das Monotonieverhalten von \(f\) in der Umgebung \(\left]-\sqrt{2}, +\infty\right[\  \setminus \{\sqrt{2}\}\). Nach dem eben bestimmten Verhalten von \(f'\) ist \(f\) monoton fallend im Intervall \(\left]-\infty, -\sqrt{2}\right[\).\\
Somit ist das Monotonieverhalten von \(f\) vollständig geklärt.\\
Damit gilt alles, was zu zeigen war. \qed \pagebreak
\section*{Aufgabe 2}
\textbf{Vor.:} Es seien \(a,b \in \mathbb R\) mit \(a < b\), sowie \(f,g : [a,b] \rightarrow \mathbb R\) stetige Funktionen, sodass \(f(a) \le g(a)\) gilt. Des Weiteren seien \(f,g\) differenzierbar auf \(\left]a,b\right[\), sodass \(f'(x) \le g'(x)\) gilt für alle \(x \in \left]a,b\right[\). \\
\textbf{Beh.:} Es gilt \(f(x) \le g(x)\) für alle \(x \in \left[a,b\right]\) \\
\textbf{Bew.:} Definiere folgende Hilfsfunktion: \(h: \left[a,b\right] \rightarrow \mathbb R, x \mapsto g(x) - f(x)\). \\
Nach der Voraussetzung gilt folgende Umformung: 
\begin{flalign*}
    & f(a) \le g(a) & \text{| \(- f(a)\)} \\
    & \Longleftrightarrow 0 \le g(a) - f(a) & \text{| Einsetzen} \\
    & \Longleftrightarrow 0 \le h(a) & \\
    & \Longleftrightarrow h(a) \ge 0
\end{flalign*}
Des Weiteren gilt nach der Voraussetzung folgendes für alle \(x \in \left]a,b\right[\) :
\begin{flalign*}
    & f'(x) \le g'(x) & \text{| \(- f'(x)\)} \\
    & \Longleftrightarrow 0 \le g'(x) - f'(x) & \text{| Kombinationssatz der Differenzierbarkeit} \\
    & \Longleftrightarrow 0 \le h'(x) & \\
    & \Longleftrightarrow h'(x) \ge 0
\end{flalign*}
Damit ist \(h\) monoton steigend. Also gilt für alle \(x \in \left[a,b\right]\) : \(h(x) \ge h(a) \ge 0\). \pagebreak \\
Dann gilt \(h(x) \ge 0\), und damit folgendes:
\begin{flalign*}
    & g(x) - f(x) \ge 0 & \\
    & \Longleftrightarrow g(x) \ge f(x) & \\
    & \Longleftrightarrow f(x) \le g(x)
\end{flalign*}
Was zu zeigen war. \qed
\section*{Aufgabe 3}
\textbf{Vor.:} \((x_n)_n \in \mathcal{S}(\mathbb{R})\) beliebig, sodass \(\lim_{n}(x_{n+1} - x_n ) = 0\) gilt. \\
\textbf{Beh.:} \(\lim_{n} x_n \in \mathbb{R}\) \\
\textbf{Bew.:} Die Aussage zeigen wir indirekt, dass heißt, wir werden in den Fällen \(\lim_n x_n \in \{+\infty, -\infty\}\) und \(\lim_n x_n \not\in \mathbb{R}\) die Aussage \(\lim_{n} (x_{n+1} - x_n) = 0\) ad absurdum führen. \\
Betrachte folgende Überlegung: \\
Wenn \(\lim_{n} x_n \in \{+\infty, -\infty\}\) gilt, bedeutet das, dass ab einen bestimmten Punkt, \((x_n)_n\) streng monoton steigend bzw. streng monoton fallend sein muss. Dies folgt aus Satz 2.20c) bzw. Satz 2.20d). Dies bedeutet aber auch, dass zwischen einer Folgenkomponente und einer weiteren Folgenkomponente bis ins Unendliche ab diesen Punkt eine positive bzw. negative Differenz existiert. \\
Diese Differenz kann kleiner oder größer werden, aber sie wird wegen der Monotonie keinen Vorzeichenwechsel mehr vollbringen oder wird sie jemals Null sein. Würde sich die Differenz an Null annähern, konvergiere \((x_n)_n\) in \(\mathbb R\), was ein Widerspruch zur Annahme wäre. \\
Somit kann im Fall \(\lim_{n} x_n \in \{+\infty, -\infty\}\), \(\lim_{n}(x_{n+1} - x_n ) = 0\) nicht gelten. \\
Im Fall, dass \((x_n)_n\) unbestimmt divergiert, kann die Differenz zwischen \(x_n\) und \(x_{n+1}\) nicht bestimmt gegen null konvergieren, da sonst eine Umgebung von null existieren würde, in welcher sich die Differenz aller Folgepartner ab eines bestimmten Punktes aufhalten würde. Dies impliziert die Existenz eines Limes der Folge \((x_n)_n\), da somit eine Kugel mit Radius der Differenz um einen Punkt existieren würde, welche jegliche Folgekomponenten nicht mehr verlassen könnten ab einem bestimmten \(n_0\). Somit wäre dann der Limes der Folge \((x_n)_n\) genau der eben erwähnte Punkt, nach der Definition des Limes. \\
Dies ist ein Widerspruch zur Annahme, dass \((x_n)_n\) unbestimmt divergiert, also kann \((x_n)_n\) nicht unbestimmt divergieren. \\
Insgesamt sind nun beide Fälle des Oberfalls \((x_n)_n\) divergiert geklärt, also kann \((x_n)_n\) nur konvergieren, was zu zeigen war. \qed
\end{document}