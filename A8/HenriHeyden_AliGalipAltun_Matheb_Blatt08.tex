\documentclass[12pt, a4paper]{article}

\usepackage[ngerman]{babel} 
\usepackage[T1]{fontenc}
\usepackage{amsfonts} 
\usepackage{setspace}
\usepackage{amsmath}
\usepackage{amssymb}
\usepackage{titling}


\newcommand*{\qed}{\null\nobreak\hfill\ensuremath{\square}}
\newcommand*{\puffer}{\text{ }\text{ }\text{ }\text{ }}
\newcommand*{\gedanke}{\textbf{-- }}


\pagestyle{plain}
\allowdisplaybreaks

\setlength{\droptitle}{-11em}

\title{Mathematik für die Informatik B - Hausaufgabenserie 8}
\author{Henri Heyden, Ali Galip Altun \\ \small stu240825, stu242631}
\date{}


\begin{document}
\maketitle

\doublespacing

\section*{Aufgabe 1}
\textbf{Vor.:} \(f: \mathbb R \setminus \{\pm \sqrt{2}\} \rightarrow \mathbb R, x \mapsto \frac{3x^2+x-8}{x^2-2}\), \\
\(I_1 := \left] -\infty, -\sqrt{2} \right[\), \\
\(I_2 := \left] -\sqrt{2}, 2 - \sqrt{2}\right[\), \\
\(I_3 := [2 - \sqrt{2}, \sqrt{2}[\), \\
\(I_4 := ]\sqrt{2}, 2+\sqrt{2}]\), \\
\(I_5 := [2+\sqrt{2}, +\infty[\), \\
\textbf{Beh.:} \(\text{LMAX}(f) = \{2 + \sqrt{2}\}, \text{ LMIN}(f) = \{2 - \sqrt{2}\}\), \\
\(f\) ist streng monoton fallend in \(I_1\), \\
streng monoton fallend in \(I_2\),\\
streng monoton steigend in \(I_3\), \\
streng monoton steigend in \(I_4\) und \\
streng monoton fallend in \(I_5\).\\
\textbf{Bew.:} Nach Vorlesung ist \(f\) differenzierbar in ihrer Domain mit der Ableitung (Quotientenregel): \(f'(x) = \frac{(6x + 1) \cdot (x^2-2) - (3x^2+x-8)\cdot 2x}{(x^2-2)^2}\). Wenn man dies kürzt indem man alle Klammern auflöst und vereinfacht, ergibt sich: \(f'(x) = \frac{-x^2 + 4x - 2}{(x^2 - 2)^2}\). Nach Satz 3.48 müssen Nullstellen von \(f'(x)\) existieren, sodass \(f\) Extremstellen haben kann, also suchen wir Lösungen für \(f'(x) = 0\).\\
Betrachte folgende Umformung: \pagebreak
\begin{flalign*}
    & 0 = f'(x) & \text{} \\
    & 0 = \frac{-x^2 + 4x - 2}{(x^2 - 2)^2} & \text{| \(\cdot (x^2 - 2)^2\), \(x \ne \pm \sqrt{2}\)} \\
    & \Longleftrightarrow 0 = {-x^2 + 4x - 2} & \text{| Lösung von einer quadratischen Gleichung} \\
    & \Longleftrightarrow x_{1,2} = \frac{-4}{-2} \pm \frac{\sqrt{16-8}}{-2} & \text{| Auswerten} \\
    & x_{1,2} = 2 \mp \sqrt{2} & 
\end{flalign*}
Nun wissen wir, dass die Stellen \(x = 2 \pm \sqrt{2}\) möglicherweise Extremstellen sein können. Gleichzeitig sind dies die einzigen Stellen, die Extremstellen sein können, aufgrund vom Satz 3.48. \\
% Da \(f\) vollständig differenzierbar ist, und somit auch vollständig stetig ist in ihrer Domain, betrachten wir den Differenzialwert von \(f'\) an den Stellen \(f(2 \pm \sqrt{2})\) um herauszufinden, ob \(f'\) einen Vorzeichenwechsel um diese Stelle hat, und wenn, was für ein VZW vorliegt. \\
% \(f'\) ist vollständig differenzierbar in jedem Punkt außer \(\pm\sqrt{2}\) (nach Eigenschaften von rationalen Funktionen) mit der Ableitung \[f''(x) = \frac{(-2x+4) \cdot (x^2-2)^2 - (-x^2 + 4x -2) \cdot (4x^3-8x)}{(x^2-2)^4}\]
% Dies folgt aus der Quotientenregel und den Potenzgesetzen. \\
% Beobachte: \(f''(2 + \sqrt{2}) < 0\) und \(f''(2 - \sqrt{2}) > 0\). \\
% Somit hat \(f'\) einen VZW von \(+ \rightarrow -\) und von \(- \rightarrow +\) in den Stellen \(2 + \sqrt{2}\) bzw. \(2 - \sqrt{2}\). \\
% Hieraus folgt, dass\dots \pagebreak \\
% \textbf{1.:} \(f\) ein lokales Maximum in \(2 + \sqrt{2}\) besitzt, da die Umgebung \(]\sqrt{2}, +\infty[\) nur kleinere Werte enthält, was daraus folgt, dass \(f\) streng monoton steigend ist im Intervall \(]\sqrt{2}, 2+\sqrt{2}]\) und streng monoton fallend ist im Intervall \([2+\sqrt{2}, +\infty[\). \\
% Dies beides folgt aus den Eigenschaften von \(f'\), die wir gerade bestimmt hatten. \\
% \textbf{2.:} \(f\) ein lokales Minimum in \(2 - \sqrt{2}\) besitzt, da die Umgebung \(]- \sqrt{2}, + \sqrt{2}\) nur größere Werte enthält, was daraus folgt, dass \(f\) streng monoton fallend ist im Intervall \(] - \sqrt{2}, 2 - \sqrt{2}]\) und streng monoton steigend ist im Intervall \([2 - \sqrt{2}, \sqrt{2}[\). \\
% Dies beides folgt aus den Eigenschaften von \(f'\), die wir gerade bestimmt hatten. \\
% Nun ist das Extremverhalten von \(f\) geklärt.\\
% Des Weiteren sind wir nun aufgeklärt über das Monotonieverhalten von \(f\) in der Umgebung \(\left]-\sqrt{2}, +\infty\right[\  \setminus \{\sqrt{2}\}\). Nach dem eben bestimmten Verhalten von \(f'\) ist \(f\) monoton fallend im Intervall \(\left]-\infty, -\sqrt{2}\right[\).\\
% Somit ist das Monotonieverhalten von \(f\) vollständig geklärt.\\
Es gilt: \[-3 \in I_1 \wedge f'(-3) = -\frac{23}{49} < 0\]
\[-1 \in I_2 \wedge f'(-1) = -\frac{7}{1} < 0\]
\[1 \in I_3 \wedge f'(1) = \frac{1}{1} > 0\]
\[2 \in I_4 \wedge f'(2) = \frac{2}{4} > 0\]
\[4 \in I_5 \wedge f'(4) = -\frac{2}{144} < 0\]
Nach Einsatztechnik folgt die Behauptung. \\
Damit gilt alles, was zu zeigen war. \qed \pagebreak
\section*{Aufgabe 2}
\textbf{Vor.:} Es seien \(a,b \in \mathbb R\) mit \(a < b\), sowie \(f,g : [a,b] \rightarrow \mathbb R\) stetige Funktionen, sodass \(f(a) \le g(a)\) gilt. Des Weiteren seien \(f,g\) differenzierbar auf \(\left]a,b\right[\), sodass \(f'(x) \le g'(x)\) gilt für alle \(x \in \left]a,b\right[\). \\
\textbf{Beh.:} Es gilt \(f(x) \le g(x)\) für alle \(x \in \left[a,b\right]\) \\
\textbf{Bew.:} Definiere folgende Hilfsfunktion: \(h: \left[a,b\right] \rightarrow \mathbb R, x \mapsto g(x) - f(x)\). \\
Nach der Voraussetzung gilt folgende Umformung: 
\begin{flalign*}
    & f(a) \le g(a) & \text{| \(- f(a)\)} \\
    & \Longleftrightarrow 0 \le g(a) - f(a) & \text{| Einsetzen} \\
    & \Longleftrightarrow 0 \le h(a) & \\
    & \Longleftrightarrow h(a) \ge 0
\end{flalign*}
Des Weiteren gilt nach der Voraussetzung folgendes für alle \(x \in \left]a,b\right[\) :
\begin{flalign*}
    & f'(x) \le g'(x) & \text{| \(- f'(x)\)} \\
    & \Longleftrightarrow 0 \le g'(x) - f'(x) & \text{| Kombinationssatz der Differenzierbarkeit} \\
    & \Longleftrightarrow 0 \le h'(x) & \\
    & \Longleftrightarrow h'(x) \ge 0
\end{flalign*}
Damit ist \(h\) monoton steigend. Also gilt für alle \(x \in \left[a,b\right]\) : \(h(x) \ge h(a) \ge 0\). \pagebreak \\
Dann gilt \(h(x) \ge 0\), und damit folgendes:
\begin{flalign*}
    & g(x) - f(x) \ge 0 & \\
    & \Longleftrightarrow g(x) \ge f(x) & \\
    & \Longleftrightarrow f(x) \le g(x)
\end{flalign*}
Was zu zeigen war. \qed
\section*{Aufgabe 3}
\textbf{Vor.:} \((x_n)_n \in \mathcal{S}(\mathbb{R}), x_n := \frac{1}{n}\). Sei \((a_n)_n\) die Reihe über \((x_n)_n\) \\
\textbf{Beh.:} Die Aussage gilt nicht. Wir zeigen, dass für \((a_n)_n\) folgendes gilt: \(\lim_n(a_{n+1} - a_n) = 0 \wedge \lim_n(a_n) \not\in \mathbb R\) \\
\textbf{Bew.:} \(\lim_n(a_n) = +\infty \not\in \mathbb R\) folgt aus Satz 2.54.\\
Betrachte folgende Umformung:
\vspace*{-0.5cm}
\setlength{\jot}{5pt}
\begin{flalign*}
    & \puffer \lim_n (a_{n+1} - a_{n}) & \\
    & \text{ } = \lim_n \left(\sum_{i = 1}^{n + 1} \frac{1}{i} - \sum_{i = 1}^{n} \frac{1}{i}\right) & \\
    & \text{ } = \lim_n \left(\frac{1}{n} + \sum_{i = 1}^{n} \frac{1}{i} - \sum_{i = 1}^{n} \frac{1}{i}\right) & \\
    & \text{ } = \lim_n \frac{1}{n} \\
    & \text{ } = 0
\end{flalign*}
Damit gilt, was zu zeigen war. \qed
\end{document}