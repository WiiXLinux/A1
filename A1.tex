\documentclass[12pt, a4paper]{article}

\usepackage[ngerman]{babel} 
\usepackage[T1]{fontenc}
\usepackage{amsfonts} 
\usepackage{setspace}
\usepackage{amsmath}
\usepackage{amssymb}

\newcommand*{\qed}{\null\nobreak\hfill\ensuremath{\square}}

\pagestyle{plain}

\title{Mathematik für die Informatik B - Hausaufgabenserie 1}
\author{Florian Schlösser, Henri Heyden \\ \small stu240349, stu240825}
\date{}

\begin{document}
\maketitle

\doublespacing
\textbf{Aufgabe 1} \\
Es ist zu zeigen, dass $ \forall x,y > 0: \exists n \in \mathbb N: nx > y$ eine wahre Aussage ist. \\
Wir werden dies mithilfe eines direkten Beweises zeigen. \\
Vorausgesetzt ist, dass $n \in \mathbb N, x,y\in \mathbb R_{>0}$ gelten.\\
Des weiteren definieren wir das Aufrunden als Operation für nicht-negative reelle Zahlen: $\lceil \cdot \rceil : \mathbb R_{>0} \rightarrow \mathbb N, x \mapsto min\{n \in \mathbb N | n \ge x\}  $\\
Wir setzen $n := \lceil\frac{2y}{x}\rceil$. Da $n \in \mathbb N$ gilt, wird der Term aufgerundet.\\
Dann gilt folgendes:

\vspace{-1cm}
\begin{align*}
	& nx & \text{Einsetzen in n}\\
	& = \left\lceil \frac{2y}{x} \right\rceil x & \text{Def. Aufrundung}\\
	& \ge \frac{2y}{x}x &\text{$\frac 1 x$ invers zu $x$} \\
	& = 2y & \text{Satz 1.5 (21)}\\
	& > y
\end{align*}
\hspace{-0.15cm}
Also gilt: $\left\lceil \frac{2y}{x} \right\rceil x = nx > y$ \\
Hierdurch wurde gezeigt, dass das Archimedische Axiom gilt. \qed 

\textbf{Aufgabe 2} \\
Es ist zu zeigen, dass $\forall a,b \in \mathbb R: a < b \Rightarrow \exists x \in \mathbb R \setminus \mathbb Q : a < x < b$\\
Wir werden dies mithilfe eines direkten Beweises zeigen. \\
Hier ist vorauszusetzen, dass $a,b \in \mathbb R \wedge x \in \mathbb R \setminus \mathbb Q \wedge a < b$ gilt. \\
Da für alle zwei Zahlen aus $\mathbb R$ gilt, dass wenn sie nicht gleich sind, es unendlich viele Zahlen zwischen ihnen gibt, folgern wir, dass es unendlich Zahlen aus $\mathbb R$ gibt, die zwischen $a$ und $b$ liegen. \\
Nun werden wir zeigen, dass es auch unendlich Zahlen $x$ der gesuchten Eigentschaft auch in $(a,b) \setminus \mathbb Q$ gibt. \\
Wir definieren $x := a + \alpha \sqrt{2}$, wobei $\alpha$ eine rationale Zahl ist, dessen Intervall wir noch bestimmen werden.\\
Des weiteren gilt, wie in MatheA bewiesen, $\sqrt{2} \in \mathbb R \setminus \mathbb Q$. Somit gilt: $x \in \mathbb R \setminus \mathbb Q$ aufgrund der Definition von $x$. \\
Damit $a < x < b$ gilt, müssen wir $\alpha$ so wählen, sodass $a < a + \alpha \sqrt{2} < b$ gilt. \\
Dazu formen wir die Ungleichungen $a < a + \alpha \sqrt{2}$ und $a + \alpha \sqrt{2} < b$ nach $\alpha$ um:

\textbf{(1.)}
\vspace{-1cm}
\begin{align*}
	& a < a + \alpha \sqrt{2} & \text{-a}\\
	& \Longleftrightarrow 0 < \alpha \sqrt{2} & \text{:$\sqrt{2}$}\\
	& \Longleftrightarrow 0 < \alpha
\end{align*}
\hspace{-0.15cm}
Somit ist die linke exklusive Grenze für $\alpha = 0$.\\
\\\\\\

\textbf{(2.)}
\vspace{-1cm}
\begin{align*}
	& a + \alpha \sqrt{2} < b & \text{-a}\\
	& \Longleftrightarrow \alpha \sqrt{2} < b - a & \text{:$\sqrt{2}$}\\
	& \Longleftrightarrow \alpha < \frac{b - a}{\sqrt{2}}
\end{align*}
\hspace{-0.15cm}
Damit ist die rechte exklusiv Grenze für $\alpha = \frac{b - a}{\sqrt{2}}$.\\
Insgesamt schließt sich daraus, dass $\alpha \in (0,\frac{b - a}{\sqrt{2}})_{\mathbb Q}$ gilt.\\
Nun um zu prüfen, dass unsere $x$ nun die geforderte Bedingung erfüllen, setzen wir linke und rechte Grenze in $\alpha$ und zeigen, dass dann $x = a$ und $x = b$ gelten würde:\\
\textbf{1. ($\alpha = 0$):}
\vspace{-1cm}
\begin{align*}
	& x = 0 \cdot \sqrt{2} + a &\\
	& = a &
\end{align*}

\hspace{-0.75cm}
\textbf{2. ($\alpha = \frac{b - a}{\sqrt{2}}$):}
\vspace{-1cm}
\begin{align*}
	& x = \frac{b - a}{\sqrt{2}} \cdot \sqrt{2} + a & \\
	& = b - a + a & \\
	& = b &
\end{align*}
\hspace{-0.15cm}
Somit gelte dann $x = a$ bzw. $x = b$, wodurch wir wissen, dass wenn $\alpha \in (0,\frac{b - a}{\sqrt{2}})_{\mathbb Q}$ gilt, dass dann $x \in (a,b) \Longleftrightarrow a < x < b$ gilt.
Da wir schon gezeigt hatten, dass für diese $x$, $x \in \mathbb R \setminus \mathbb Q$ gilt, ist somit $x \in \mathbb R \setminus \mathbb Q \wedge a < x < b$ wahr. \\
Damit wurde $\forall a,b \in \mathbb R: a < b \Rightarrow \exists x \in \mathbb R \setminus \mathbb Q : a < x < b$ gezeigt. \qed
\end{document}
