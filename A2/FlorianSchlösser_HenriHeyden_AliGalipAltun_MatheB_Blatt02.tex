\documentclass[12pt, a4paper]{article}

\usepackage[ngerman]{babel} 
\usepackage[T1]{fontenc}
\usepackage{amsfonts} 
\usepackage{setspace}
\usepackage{amsmath}
\usepackage{amssymb}

\newcommand*{\qed}{\null\nobreak\hfill\ensuremath{\square}}

\pagestyle{plain}

\title{Mathematik für die Informatik B - Hausaufgabenserie 2}
\author{Florian Schlösser, Henri Heyden, Ali Galip Altun \\ \small stu240349, stu240825, stu242631}
\date{}

\begin{document}
\maketitle

\doublespacing
\section*{Aufgabe 1}
Es ist zu zeigen, dass $\lim_{n}\frac{\sqrt{n}}{2^n} = 0$ gilt. \\
Hierfür werden wir den Sandwichsatz aka. Satz 2.20, anwenden. \\
Wir nehmen an, dass $n \in \mathbb{N}$ gilt und, dass alle in dieser Bearbeitung erwähnten Folgen wohldefiniert sind. \\
Für den Sandwichsatz definieren wir die folgenden Folgen: \\ $(a_n)_n := \left(\frac{n}{2^n}\right)_n$, $(b_n)_n := \left(\frac{1}{2^n}\right)_n$, $(x_n)_n := \left(\frac{\sqrt{n}}{2^n}\right)_n$.\\
Nun werden wir zeigen, dass $\lim_{n}a_n = \lim_{n}b_n = 0$ gilt. \\
In der Präsenzübung wurde gezeigt, dass \textbf{(A)} $\lim_{n}a_n = 0$ gilt, \\
weswegen wir nur zeigen müssen, dass \textbf{(B)} $\lim_{n}b_n = 0$ gilt damit wir den ersten Teil des Sandwichsatzes erfüllen. \\
Beobachte, dass $\forall n \in \mathbb N: 2^n = |2^n|$ gilt, da $2^n$ für alle $n \in \mathbb N$ monoton steigend ist, somit ist nach Satz 2.24 die Aussage \textbf{B} äquivalent zu $\lim_{n}\frac{1}{b_n} = +\infty$ \\
Um $\lim_{n}\frac{1}{b_n} = \lim_{n}2^n = +\infty$ zu zeigen, wenden wir Satz 2.10.c an: \\
$\lim_{n}2^n = +\infty \Longleftrightarrow \forall r: \exists n_0: \forall n \ge n_0: r < 2^n$. \\
Wir wissen allgemein, dass $n \ge n_0 \Rightarrow 2^n \ge 2^{n_0}$ gilt, da wie erwähnt $2^n$ monoton steigend ist.
Wähle $r$ beliebig, dann setzen wir $n_0 := |r| + 1$ und wir untersuchen die folgenden Fälle:\\
Hierfür setzen wir $d:= n - n_0$, beobachte, $d \ge 0$ nach den gegebenen Definitionen für $n$ und $n_0$. \\
\textbf{(1.) $r < 0$:} \\
Dann gilt: $r < 0 < b_n$, da der minimale Wert für $b_n$, 1 ist. \\
\textbf{(2.) $r \ge 0$:} \\
Dann gilt: $r < b_n = 2^{n} = 2^{n_0 + d} = 2^{|r| + 1 + d} = 2^{r + 1 + d} = 2 \cdot 2^r \cdot 2^d$, \\
da $2 \cdot 2^d \ge 2$ und $r \le 2^r$ \\
Somit wurde $\lim_{n}\frac{1}{b_n} = +\infty$ gezeigt und damit auch (erinnere \textbf{A}), dass $\lim_{n}a_n = \lim_{n}b_n = 0$ gilt.\\
Nach dem Sandwichsatz müssen wir somit nur noch zeigen, dass \\
$\forall n \in \mathbb N: b_n \le x_n \le a_n$ gilt. Betrachte folgende Umformung:
\vspace{-0.5cm}
\begin{align*}
    & b_n \le x_n \le a_n & \\
    & \Longleftrightarrow \frac{1}{2^n} \le \frac{\sqrt{n}}{2^n} \le \frac{n}{2^n} &| \cdot 2^n \\
    & \Longleftrightarrow 1 \le \sqrt{n} \le n &
\end{align*}
Dies ist wahr für $n \ge 1$, was nach der Beobachtung 2.21 im Skript bedeutet, dass der Sandwichsatz vollständig anwendbar ist, wonach \\
$\lim_{n}\frac{\sqrt{n}}{2^n} = \lim_{n}a_n = \lim_{n}b_n = 0$ gilt. \qed \\



\section*{Aufgabe 2}
Es ist zu zeigen, dass $\left(\frac{1}{n}\sum_{k = 1}^{n}x_k\right)_{n \ge 1}$ konvergiert, wenn wir wissen, dass $(x_k)_{k \ge 1}$ konvergiert. Das heißt es existiert ein Limies der Folge $\left(\frac{1}{n}\sum_{k = 1}^{n}x_k\right)_{n \ge 1}$,
welche wir folgend $m$ nennen werden, der in den reelen Zahlen liegt, wenn dieses Kriterium für die Folge $(x_k)_{k \ge 1}$ gilt. \\
Da $(x_k)_{k \ge 1}$ konvergiert, wissen wir nach der Definiton der Konvergenz (2.1.5), dass demnach für diese Folge ein reeler Grenzwert existiert, den wir folgend mit $y$ bezeichnen. Wir schreiben also: $y := \lim_{k}x_k$. \\
Wir werden zeigen, dass $y$ unser gesuchter Limes der Folge $m$ ist. Hiermit würden wir wissen, dass damit der Grenzwert von $m$ auch in $\mathbb R$ liegt, wodurch nach der Definiton der Konvergenz die Folge $m$ konvergiert. \\
Nach Satz 2.10.b ist hiermit zu zeigen: $\forall \epsilon > 0: \exists n_0: \forall n \ge n_0: |m_n - y| < \epsilon$.
Wähle $\epsilon$ beliebig größer $0$. \\

% Das kommt später ;)
% Setze $n_0 := \left\lceil{2t \cdot \frac{a - 1}{\epsilon}}\right\rceil$


\end{document}