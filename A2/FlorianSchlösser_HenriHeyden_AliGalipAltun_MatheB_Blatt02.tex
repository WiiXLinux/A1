\documentclass[12pt, a4paper]{article}

\usepackage[ngerman]{babel} 
\usepackage[T1]{fontenc}
\usepackage{amsfonts} 
\usepackage{setspace}
\usepackage{amsmath}
\usepackage{amssymb}

\newcommand*{\qed}{\null\nobreak\hfill\ensuremath{\square}}
\newcommand*{\puffer}{\text{ }\text{ }\text{ }\text{ }}

\pagestyle{plain}
\allowdisplaybreaks

\title{Mathematik für die Informatik B - Hausaufgabenserie 2}
\author{Florian Schlösser, Henri Heyden, Ali Galip Altun \\ \small stu240349, stu240825, stu242631}
\date{}


\begin{document}
\maketitle


\doublespacing
\section*{Aufgabe 1}
Es ist zu zeigen, dass $\lim_{n}\frac{\sqrt{n}}{2^n} = 0$ gilt. \\
Hierfür werden wir den Sandwichsatz aka. Satz 2.20, anwenden. \\
Wir nehmen an, dass $n \in \mathbb{N}$ gilt und, dass alle in dieser Bearbeitung erwähnten Folgen wohldefiniert sind. \\
Für den Sandwichsatz definieren wir die folgenden Folgen: \\ $(a_n)_n := \left(\frac{n}{2^n}\right)_n$, $(b_n)_n := \left(\frac{1}{2^n}\right)_n$, $(x_n)_n := \left(\frac{\sqrt{n}}{2^n}\right)_n$.\\
Nun werden wir zeigen, dass $\lim_{n}a_n = \lim_{n}b_n = 0$ gilt. \\
In der Präsenzübung wurde gezeigt, dass \textbf{(A)} $\lim_{n}a_n = 0$ gilt, \\
weswegen wir nur zeigen müssen, dass \textbf{(B)} $\lim_{n}b_n = 0$ gilt damit wir den ersten Teil des Sandwichsatzes erfüllen. \\
Beobachte, dass $\forall n \in \mathbb N: 2^n = |2^n|$ gilt, da $2^n$ für alle $n \in \mathbb N$ monoton steigend ist, somit ist nach Satz 2.24 die Aussage \textbf{B} äquivalent zu $\lim_{n}\frac{1}{b_n} = +\infty$ \\
Um $\lim_{n}\frac{1}{b_n} = \lim_{n}2^n = +\infty$ zu zeigen, wenden wir Satz 2.10.c an: \\
$\lim_{n}2^n = +\infty \Longleftrightarrow \forall r: \exists n_0: \forall n \ge n_0: r < 2^n$. \\
Wir wissen allgemein, dass $n \ge n_0 \Rightarrow 2^n \ge 2^{n_0}$ gilt, da wie erwähnt $2^n$ monoton steigend ist.
Wähle $r$ beliebig, dann setzen wir $n_0 := |r| + 1$ und wir untersuchen die folgenden Fälle:\\
Hierfür setzen wir $d:= n - n_0$, beobachte, $d \ge 0$ nach den gegebenen Definitionen für $n$ und $n_0$. \\
\textbf{(1.) $r < 0$:} \\
Dann gilt: $r < 0 < b_n$, da der minimale Wert für $b_n$, 1 ist. \\
\textbf{(2.) $r \ge 0$:} \\
Dann gilt: $r < b_n = 2^{n} = 2^{n_0 + d} = 2^{|r| + 1 + d} = 2^{r + 1 + d} = 2 \cdot 2^r \cdot 2^d$, \\
da $2 \cdot 2^d \ge 2$ und $r \le 2^r$ \\
Somit wurde $\lim_{n}\frac{1}{b_n} = +\infty$ gezeigt und damit auch (erinnere \textbf{A}), dass $\lim_{n}a_n = \lim_{n}b_n = 0$ gilt.\\
Nach dem Sandwichsatz müssen wir somit nur noch zeigen, dass \\
$\forall n \in \mathbb N: b_n \le x_n \le a_n$ gilt. Betrachte folgende Umformung:
\vspace{-0.5cm}
\begin{align*}
    & b_n \le x_n \le a_n & \\
    & \Longleftrightarrow \frac{1}{2^n} \le \frac{\sqrt{n}}{2^n} \le \frac{n}{2^n} &| \cdot 2^n \\
    & \Longleftrightarrow 1 \le \sqrt{n} \le n &
\end{align*}
Dies ist wahr für $n \ge 1$, was nach der Beobachtung 2.21 im Skript bedeutet, dass der Sandwichsatz vollständig anwendbar ist, wonach \\
$\lim_{n}\frac{\sqrt{n}}{2^n} = \lim_{n}a_n = \lim_{n}b_n = 0$ gilt. \qed \\



\section*{Aufgabe 2}
Es ist zu zeigen, dass $\left(\frac{1}{n}\sum_{k = 1}^{n}x_k\right)_{n \ge 1}$ konvergiert, wenn wir wissen, dass $(x_k)_{k \ge 1}$ konvergiert. Das heißt es existiert ein Limies der Folge $\left(\frac{1}{n}\sum_{k = 1}^{n}x_k\right)_{n \ge 1}$,
welche wir folgend $m_n$ nennen werden, der in den reelen Zahlen liegt, wenn dieses Kriterium für die Folge $(x_k)_{k \ge 1}$ gilt. \\
Da $(x_k)_{k \ge 1}$ konvergiert, wissen wir nach der Definiton der Konvergenz (2.1.5), dass demnach für diese Folge ein reeler Grenzwert existiert, den wir folgend mit $y$ bezeichnen. Wir schreiben also: $y := \lim_{k}x_k$. \\
Wir werden zeigen, dass $y$ unser gesuchter Limes der Folge $m_n$ ist. Hiermit würden wir wissen, dass damit der Grenzwert von $m_n$ auch in $\mathbb R$ liegt, wodurch nach der Definiton der Konvergenz die Folge $m_n$ konvergiert. \\
Nach Satz 2.10.b ist hiermit zu zeigen: $\forall \epsilon > 0: \exists n_0: \forall n \ge n_0: |m_n - y| < \epsilon$.
Wähle $\epsilon$ beliebig größer $0$. Setze $n_0 := \left\lceil\frac{n_1 \cdot t}{\epsilon}\right\rceil$. Wir werden zeigen, wie wir auf dieses $n_0$ kommen, und warum es sich eignet. \\
Bei der folgenden Umformung ist zu beachten, dass mit $n_1$ der Mindestwertindex der Folge $(x_k)_k$ ist, sodass diese das Epsilon Kriterium erfüllt:
\vspace{-0.5cm}
\begin{align*}
    & \puffer |m_n - y| & \text{| Einsetzen in $m_n$} \\
    & = \left|-y + \frac{1}{n} \sum_{k=1}^{n}x_k\right| &  \text{| Def. Inverses} \\
    & = \left|n \cdot \frac{1}{n} \cdot (-y) + \frac{1}{n} \sum_{k=1}^{n}x_k\right| & \text{| }n \cdot y = \sum_{k=1}^{n} y\\
    & = \left|\frac{1}{n} \cdot \sum_{k=1}^{n}(x_k - y)\right| & \text{| $\frac{1}{n}$ positiv, da $n > 0$} \\
    & = \frac{1}{n} \cdot \left| \sum_{k=1}^{n}(x_k - y)\right| & \text{| Dreiecksungleichung (Satz 2.2.6)} \\
    & \le \frac{1}{n} \cdot \sum_{k=1}^{n}|x_k - y| & \text{| Spalte die Summe am Punkt $n_1$} \\
    & = \frac{1}{n} \cdot \sum_{k=1}^{n_1 -1 }|x_k - y| + \frac{1}{n} \cdot \sum_{k=n_1}^{n}|x_k - y| & \text{| beobachte $|x_k - y| < \epsilon$} \\
    & < \frac{1}{n} \cdot \sum_{k=1}^{n_1 -1 }|x_k - y| + \frac{1}{n} \cdot \sum_{k=n_1}^{n}\epsilon & \\
    & \le \frac{1}{n} \cdot \sum_{k=1}^{n_1 -1 }|x_k - y| + \frac{1}{n} \cdot \sum_{k=1}^{n}\epsilon & \\
    & \le \frac{1}{n} \cdot \sum_{k=1}^{n_1 -1 }|x_k - y| + \epsilon &
\end{align*}
Beobachte, dass die Summe aus endlich vielen Summanden besteht. Von diesen Summanden existiert ein Maximum.\\
Wir nennen es $t$ mit $t := max\big\{|x_\alpha - y| \big| \alpha \in [1,n_1]_\mathbb{N}\big\}$, dann gilt:
\begin{align*}
    & \puffer \frac{1}{n} \cdot \sum_{k=1}^{n_1 -1 }|x_k - y| + \epsilon & \text{| Setze t ein und erweitere die Summe bis auf $n_1$ statt $n_1-1$} \\
    & \le \frac{1}{n} \cdot \sum_{k=1}^{n_1}t + \epsilon & \text{| Löse die Summe auf} \\
    & = \frac{n_1 \cdot t}{n} + \epsilon
\end{align*}
Wir werden nun zeigen, dass $\frac{n_1 \cdot t}{n} \le \epsilon$ gilt, wenn $n_0 := \left\lceil\frac{n_1 \cdot t}{\epsilon}\right\rceil$ gesetzt ist. Wir Betrachten $n := n_0$:
\begin{align*}
    & \puffer \frac{n_1 \cdot t}{n} & n_0 = n \\
    & = \frac{n_1 \cdot t}{n_0} & \text{| Setze $n_0$ ein} \\
    & = \frac{n_1 \cdot t}{\left(\frac{n_1 \cdot t}{\epsilon}\right)} & \text{| Bruchrechnung} \\
    & = \frac{\epsilon \cdot n_1 \cdot t}{n_1 \cdot t} & \text{| Bruchrechnung} \\
    & = \epsilon
\end{align*}
Da es hier gilt, gilt $\frac{n_1 \cdot t}{n} \le \epsilon$ auch für $n > n_0$, da dann $\frac{n_1 \cdot t}{n} < \frac{n_1 \cdot t}{n_0}$ gilt. \\
Insgesamt gilt damit: $\frac{n_1 \cdot t}{n} + \epsilon \le 2 \cdot \epsilon$. \\
Hierdurch haben wir gezeigt, dass $|m_n - y| < 2\cdot \epsilon$ gilt.
Nun beobachte, dass nach der Definiton des Epsilon Kriterium's  $|m_n - y| < \epsilon$ gelten soll, angenommen $\epsilon > 0$ gelte. Beobachte aber, dass angenommen $\epsilon_1 > 0$ gelte, dass dann mit $2\cdot \epsilon_1$ alle Zahlen die durch alleine $\epsilon_0 > 0$ darstellbar sind, auch durch $2\cdot \epsilon_1$ darstellbar sind, es werden nur unterschiedliche Eingabewerte für die beiden $\epsilon$ gewählt mit $\epsilon_1 = \frac{\epsilon_0}2$, angenommen $\epsilon_0 = 2\cdot \epsilon_1$ sollte gelten. \\
Durch diese Beobachtung ließe sich sogar zeigen, dass man das Epsilon Kriterium erweitern kann, sodass man nur ein positives Vielfaches von $\epsilon$ größer haben müsse (mit $\epsilon > 0 $ beliebig) für die Differenz zwischen Limes und Folge, was wir aber nicht tun werden, weil wir uns hier nur das doppelte Vielfache von $\epsilon$ interessieren und dies schon gezeigt wurde. \\
Damit erfülllt die Folge $m_n - y$ das "2-fache Epsilon Kriterium", das heißt\\
$\forall \epsilon > 0: \exists n_0: \forall n \ge n_0: |m_n - y| < 2 \cdot \epsilon$ gilt mit y als dem Limes der Folge $(x_k)_{k\ge n_1}$. \\
Damit wissen wir, dass die Folge $(m_n)_{n \ge n_0}$ konvergiert zu $y$. \qed


\end{document}