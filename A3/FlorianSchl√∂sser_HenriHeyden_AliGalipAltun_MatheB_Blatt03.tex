\documentclass[12pt, a4paper]{article}

\usepackage[ngerman]{babel} 
\usepackage[T1]{fontenc}
\usepackage{amsfonts} 
\usepackage{setspace}
\usepackage{amsmath}
\usepackage{amssymb}

\newcommand*{\qed}{\null\nobreak\hfill\ensuremath{\square}}
\newcommand*{\puffer}{\text{ }\text{ }\text{ }\text{ }}

\pagestyle{plain}
\allowdisplaybreaks

\title{Mathematik für die Informatik B - Hausaufgabenserie 3}
\author{Florian Schlösser, Henri Heyden, Ali Galip Altun \\ \small stu240349, stu240825, stu242631}
\date{}


\begin{document}
\maketitle


\doublespacing
\section*{Aufgabe 1}
Behauptung: \((x_n)_{n\ge 1}\) divergiert unbestimmt. \\
Voraussetzung: \(odd := \{2n+1 \mid n \in \mathbb N\}, even := \{2n \mid n \in \mathbb N\}\) \\
Beweis: Nehme an, \((x_n)_{n\ge 1}\) konvergiere oder divergiere bestimmt. Demnach müsse gelten, dass \((x_n)_{n\ge 1}\) einen Limes habe, den wir \(p\) nennen werden. \\
Definiere hierzu die Teilfolgen \((a_n)_{n \ge 1} := (x_{2n+1})_{n \ge 1}\) und \((b_n)_{n \ge 1} := (x_{2n})_{n \ge 1}\) von \(x\) (es ist leicht zu sehen, dass \(2n + 1\) und \(2n\) strenge monotone Abbildungen von \(n\) sind für \(n \in \mathbb N\)).\\
Beobachte, dass nach den Definitionen von \(odd\) und \(even\), \(a_n = \frac{1}{n}\) und \(b_n = n\) gelten.
Nach Satz 2.42 müssten die Limetes von den Folgen \(a\) und \(b\) gleich dem Limes von \(x\) gleichen. \\
Beobachte aber, dass \(lim_n a_n = 0 \ne lim_n b_n = +\infty\) nach Skript gilt, was uns zu einem Widerspruch der Annahme \((x_n)_{n\ge 1}\) konvergiere oder divergiere bestimmt. \\
Demnach gilt das Gegenteil, also divergiert \((x_n)_{n\ge 1}\) unbestimmt. \qed
\section*{Aufgabe 2}
Behauptung: \(\lim_{n}\sqrt[n]{c} = 1\). \\
Voraussetzung: \(c > 0\). \\
Beweis: Wir werden den Sandwichsatz (S.2.20) nutzen, um die Aussage zu zeigen. Demnach definieren wir zwei Folgen: \((a_n)_{n\ge 1} := (\frac 1 n ^ \frac 1 n)_{n \ge 1}\),\((b_n)_n := (\sqrt[n]{n})_n\) und geben der Folge, wessen Limes gesucht ist den Namen \((x_n)_n\). \\
Um den Sandwichsatz anwenden zu können werden wir Zeigen, dass für \textbf{fast} (S.2.12) alle \(n \in \mathbb{N}\) gilt, dass \(a_n \le x_n \le b_n\) und dass \(\lim_{n}a_n = \lim_{n}b_n\) wahre Aussagen sind. \\
Für die zweite Aussage wissen wir nach S.2.28: \(\lim_{n} \sqrt[n]{n} = 1\).\\
Wir werden zeigen, dass \(\lim_{n} \frac 1 n ^ \frac 1 n = 1\) gilt, betrachte folgende Umformung:
\begin{flalign*}
    & \puffer \lim_{n} \frac 1 n ^ \frac 1 n & \text{| Vereinbarung Schreibweise} \\
    & = \lim_{n} {n ^ {-1}} ^ {\frac 1 n} & \text{| Potenzgesetze} \\
    & = \lim_{n} {n} ^ {-\frac 1 n} & \text{| Potenzgesetze} \\
    & = \lim_{n} {n ^ {\frac 1 n}} ^ {-1} & \text{| Schreibweise bzw. Potenzgesetze} \\
    & = \lim_{n} {\sqrt[n]{n}} ^ {-1} & \text{| Potenzgesetze} \\
    & = \lim_{n} \frac{1}{\sqrt[n]{n}} & \text{| Präsenzaufgabe 3 und Satz 2.28} \\
    & = 1
\end{flalign*}
Damit gilt nur noch zu zeigen, dass für fast alle \(n \in \mathbb{N}\), \(a_n \le x_n \le b_n\) gilt.\\
Beobachte, dass wenn \(n = c, \sqrt[n]{n} = \sqrt[n]{c}\) gilt.\\
Demnach, wenn \(n \ge c \wedge n \ge 1\) gilt, gilt \(\sqrt[n]{n} \ge \sqrt[n]{c}\), da \(n\) steigt und \(c\) konstant ist, \(-\) im Beweis von S.2.28 im Skript wird angesprochen, dass \(\sqrt[n]{\cdot}\) streng monoton steigend ist für \(n \ge 1\) (wir nennen diese Aussage \textbf A). \\
Somit müssen wir nur noch zeigen, dass \(a_n \le x_n\) in den gegebenen Bedingungen (also, \(n \in \mathbb{N}, n \ge c \wedge n \ge 1\)) für fast alle \(n\) gilt. \\
Betrachte, dass wenn \(n \ge c\) gilt auch \(\frac{1}{n} \le c\) gilt. Aus \textbf A folgt \(\sqrt[n]{\frac{1}{n}} \le \sqrt[n]{c}\). \\
Damit beobachten wir für \(n \ge c \wedge n \ge 1\), dass \(a_n \le x_n \le b_n\) gilt, was nach S.2.12 uns genügt, um den Sandwichsatz vollständig anzuwenden. \\
Da wir \(\lim_{n}a_n = \lim_{n}b_n = 1\) schon gezeigt haben wissen wir, dass \(\lim_{n}a_n = \lim_{n}b_n = 1 = \lim_{n}x_n\) gilt. \\
Also gilt \(\lim_{n}\sqrt[n]{c} = 1\). \qed
\end{document}