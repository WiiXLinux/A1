\documentclass[12pt, a4paper]{article}

\usepackage[ngerman]{babel} 
\usepackage[T1]{fontenc}
\usepackage{amsfonts} 
\usepackage{setspace}
\usepackage{amsmath}
\usepackage{amssymb}

\newcommand*{\qed}{\null\nobreak\hfill\ensuremath{\square}}
\newcommand*{\puffer}{\text{ }\text{ }\text{ }\text{ }}

\pagestyle{plain}
\allowdisplaybreaks

\title{Mathematik für die Informatik B - Hausaufgabenserie 5}
\author{Henri Heyden, Ali Galip Altun \\ \small stu240825, stu242631}
\date{}


\begin{document}
\maketitle


\doublespacing

\section*{Aufgabe 1}
\section*{Aufgabe 2}
\textbf{Beh.:} \(M\) ist weder abgeschlossen, noch offen. \\
\textbf{Vor.:} \(M := \{\frac{1}{n} | n \in \mathbb{N}_1\}\) \\
\textbf{Bew.:} Nach Vorlesung gilt \(M = ]\ 0,1 ]\ \), da \(\frac{1}{n}\) stetig (Vorlesung) und monoton fallend (Vorlesung) ist, sowie es gilt, dass \(\frac{1}{1} = 1\) gilt und dass \(\lim_{n} \frac{1}{n} = 0\) gilt, obwohl \(\frac{1}{n} \ne 0\) gilt für alle \(n \in \mathbb{N}_1\), was in der Vorlesung auch schon angewendet wurde. \\
Nach Satz 3.2 gilt damit, dass \(M\) weder abgeschlossen noch offen ist,\\
- was zu zeigen war. \qed

\end{document}