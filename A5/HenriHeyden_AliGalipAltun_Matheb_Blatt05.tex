\documentclass[12pt, a4paper]{article}

\usepackage[ngerman]{babel} 
\usepackage[T1]{fontenc}
\usepackage{amsfonts} 
\usepackage{setspace}
\usepackage{amsmath}
\usepackage{amssymb}

\newcommand*{\qed}{\null\nobreak\hfill\ensuremath{\square}}
\newcommand*{\puffer}{\text{ }\text{ }\text{ }\text{ }}
\newcommand*{\gedanke}{\textbf{-- }}


\pagestyle{plain}
\allowdisplaybreaks

\title{Mathematik für die Informatik B - Hausaufgabenserie 5}
\author{Henri Heyden, Ali Galip Altun \\ \small stu240825, stu242631}
\date{}


\begin{document}
\maketitle

\setlength{\jot}{12pt}
\doublespacing

\section*{Aufgabe 1}
\textbf{Beh.:} Die Reihe über \(\left(c^n \cdot \left(\frac{n}{n+1}\right)^{n\cdot(n+1)} \right)_{n\ge 1}\) konvergiert absolut.\\
\textbf{Vor.:} \(n \in \mathbb{N}_{\ge 1}\) \\
\textbf{Bew.:} Wir nennen ab sofort \((y_n)_{n \ge 1} := \left(c^n \cdot \left(\frac{n}{n+1}\right)^{n\cdot(n+1)} \right)_{n\ge 1}\).\\
Betrachte folgende Vereinfachung:
\begin{flalign*}
    & c^n \cdot \left(\frac{n}{n+1}\right)^{n\cdot(n+1)} & \text{| Potenzgesetze} \\
    & = c^n \cdot \left(\left(\frac{n}{n+1}\right)^{n+1}\right)^n & \text{| \(c\) hereinziehen} \\
    & = \left(c \cdot \left(\frac{n}{n+1}\right)^{n+1}\right)^n & \text{| doppelt invertieren} \\
    & = \left(c \cdot \left(\frac{n+1}{n}\right)^{-n-1}\right)^n & \text{| Bruch vereinfachen} \\
    & = \left(c \cdot \left(1 + \frac{1}{n}\right)^{-n-1}\right)^n & \text{| doppelt invertieren} \\
    & = \left(c \cdot \left(\frac{1}{1 + \frac{1}{n}}\right)^{n+1}\right)^n
\end{flalign*}
Nach Satz 2.62 (1) konvergiert die Reihe über \(y\) absolut, wenn \(\lim_{n} \sqrt[n]{|y_n|} < 1\) gilt.
Betrachte folgende Vereinfachung:
\begin{flalign*}
    & \lim_{n} \sqrt[n]{|y_n|} & \text{| setze ein} \\
    & = \lim_{n} \sqrt[n]{\left(c \cdot \left(\frac{1}{1 + \frac{1}{n}}\right)^{n+1}\right)^n}
\end{flalign*}
\section*{Aufgabe 2}
\textbf{Beh.:} \(M\) ist weder abgeschlossen, noch offen. \\
\textbf{Vor.:} \(M := \{\frac{1}{n} | n \in \mathbb{N}_1\}\) \\
\textbf{Bew.:} Wir zeigen, dass nach Vorlesung gilt \(M = ]\ 0,1 ]\ \):\\
Da \(\frac{1}{n}\) stetig (Vorlesung) und monoton fallend (Vorlesung) ist, sowie es gilt, dass \(\frac{1}{1} = 1\) gilt und dass \(\lim_{n} \frac{1}{n} = 0\) gilt, obwohl \(\frac{1}{n} \ne 0\) gilt für alle \(n \in \mathbb{N}_1\),
gilt für alle Werte in M, dass sie kleiner oder gleich 1 sind oder größer als 0 sind, was genau \(]\ 0,1 ]\ \) erfüllt.
Nach Satz 3.2 gilt damit, dass \(M\) weder abgeschlossen noch offen ist, \gedanke was zu zeigen war. \qed\\
Wir haben tatsächlich sowas ähnliches in der Vorlesung auch schon angewendet wurde, um beispielsweise eine Folge zu finden, die seinen Limes nicht annimmt, jedoch trotzdem konvergiert.
\end{document}