\documentclass[12pt, a4paper]{article}

\usepackage[ngerman]{babel} 
\usepackage[T1]{fontenc}
\usepackage{amsfonts} 
\usepackage{setspace}
\usepackage{amsmath}
\usepackage{amssymb}
\usepackage{titling}


\newcommand*{\qed}{\null\nobreak\hfill\ensuremath{\square}}
\newcommand*{\puffer}{\text{ }\text{ }\text{ }\text{ }}
\newcommand*{\gedanke}{\textbf{-- }}


\pagestyle{plain}
\allowdisplaybreaks

\setlength{\droptitle}{-11em}
\setlength{\jot}{12pt}

\title{Mathematik für die Informatik B - Hausaufgabenserie 7}
\author{Henri Heyden, Ali Galip Altun \\ \small stu240825, stu242631}
\date{}


\begin{document}
\maketitle

\doublespacing
\section*{Aufgabe 1}
\textbf{Beh.:} Für \(n \in even_1\) ist \(f\) differenzierbar und für \(n \in odd\) ist \(f\) differenzierbar für alle \(x > 0\). \\
\textbf{Bew.:} Für ein beliebiges \(n \in \mathbb{N}_1\) gilt, dass es entweder in \(odd = \{2a+1 | a \in \mathbb{N}\}\) oder in \(even_1 = \{ 2a | 2a \ge 1 \wedge a \in \mathbb{N}\}\) liegt.\\ 
Wir schreiben \(even_1\), da \(n\) aus \(\mathbb{N}_1\) gesucht sind. \\
Hiermit können wir die folgenden Fallunterscheidungen für ein zu überprüfendes \(n\) eröffnen: \\
\textbf{A}: Es gelte \(n \in odd\) und \textbf{B}: Es gelte \(n \in even_1\). \\
Beide Fälle werden wir im Folgenden betrachten:
\subsection*{Erster Fall: A}
Gelte \(n \in odd\) ließe sich \(n\) schreiben als \(2a + 1\) für \(a \in \mathbb{N}\). Dann gelte für \(f_n\) folgendes:
\begin{flalign*}
    & f_n(x) = \sqrt{x}^n & \text{| \(n \in odd\)} \\
    & \puffer \text{ } \text{ } \text{ } = \sqrt{x}^{2a+1} & \text{| Potenzgesetze} \\
    & \puffer \text{ } \text{ } \text{ } = \sqrt{x} \cdot \sqrt{x}^{2a} & \text{| Vereinfache} \\
    & \puffer \text{ } \text{ } \text{ } = \sqrt{x} \cdot x^a \\
\end{flalign*}
Dann gilt: \(f_n(x)' = \frac{1}{2\sqrt{x}} \cdot x^a + \sqrt{x} \cdot ax^{a-1}\) nach der Ableitung der Wurzel (3.45), der Ableitung eines Polynoms (3.42) und der Produktregel (3.41). \\
Bemerke, dass gilt: \(\lim_{x \rightarrow 0} f_n(x)' = +\infty\) gilt (folgt aus den Kombinationssätzen), und somit konvergiert der Differenzenquotient von \(f_n\) nicht an der Stelle 0, womit \(f_n\) an dieser Stelle für \(n \in odd\) nicht differenzierbar ist. \\
\(x = 0\) ist die einzige Stelle bei der \(f_n\) nicht differenzierbar ist, da für \(x > 0\), \(f_n(x)' \in \mathbb{R}\) gilt, was aus den verwendeten Operationen in \(f_n(x)'\) folgt.
\subsection*{Zweiter Fall: B}
Gelte \(n \in even_1\) ließe sich \(n\) schreiben als \(2a\) für \(a \in \mathbb{N}_1\). Dann gelte für \(f_n\) folgendes:
\begin{flalign*}
    & f_n(x) = \sqrt{x}^n & \text{| \(n \in even_1\)} \\
    & \puffer \text{ } \text{ } \text{ } = \sqrt{x}^{2a} & \text{| Vereinfache} \\
    & \puffer \text{ } \text{ } \text{ } = x^a
\end{flalign*}
Nach der Ableitung eines Polynoms (3.42) ist dann \(f_n\) in jeder Stelle differenzierbar mit \(f_n(x)' = a \cdot x^{a-1}\) für \(n \in even_1\). \\ \\
Aus beiden Fallunterscheidungen folgt somit, dass für \(n \in even_1\) ist \(f\) differenzierbar und für \(n \in odd\) ist \(f\) differenzierbar für alle \(x > 0\), \\
\gedanke was zu zeigen war. \qed
\section*{Aufgabe 3}
\textbf{Beh.:} Die Aussage aus der Aufgabenstellung gilt nicht, das heißt es existieren \(a,b \in \mathbb{R}\) und \(f: [a,+\infty] \rightarrow [b,+\infty]\) so, dass \(f\) differenzierbar und streng monoton steigend ist und gleichzeitig \(\lim_{x \rightarrow +\infty} f(x) \ne +\infty\) gilt. \\
\textbf{Vor.:} \(a = b = 0\), \(f: [a,+\infty] \rightarrow [b,+\infty], x \mapsto \frac{x}{x+1}\) \\
\textbf{Bew.:} Nach Satz 3.42 können wir eine rationale Funktion wie unser \(f\) ableiten. Hierfür wenden wir ein mal den Quotenentensatz an und bekommen \(f(x)' = \frac{1}{(x+1)^2}\). Dies macht \(f\) differenzierbar in jedem Punkt, denn schließlich gilt für jedes \(x\) aus unserer Domain (\(x \ge 0\)), dass der untere Term der Ableitung nicht 0 werden kann und sonst nimmt die Ableitung auch nur Werte aus \(\mathbb{R}\) an.\\
Nun zeigen wir noch, dass \(f\) auch streng monoton steigend ist.\\
Nach dem Monotoniekriterum in Satz 3.53 zeigen wir, dass für \(f\) im Intervall ihrer Domain ihre Ableitung nur Positive Werte annimmt. \\
Wir zeigen also: \(\forall x \ge 0: f(x)' > 0\). Betrachte folgende Beobachtung: \\
Gilt \(x \ge 0\), dann gilt \(x + 1 > 0\). Dann folgt aus den Regeln in angeordneten Körpern, dass \((x + 1)^2 > 0\) gilt und daraus, dass \(\frac{1}{(x+1)^2} > 0\) gilt. \\
Somit gilt: \(\forall x \ge 0: f(x)' > 0\), also ist \(f\) streng monoton steigend. \\
Nun haben wir alle Voraussetzungen (der Aufgabenstellung) erfüllt, also werden wir zeigen, dass \(\lim_{x \rightarrow +\infty} f(x) \ne +\infty\) für unseres \(f\) gilt. \pagebreak \\
Betrachte somit folgende Auflösung des Limetes\footnote{Die Gleichheit beim ersten Schritt sieht man leicht, indem man beide Seiten von \(\frac{x}{x+1} = 1 - \frac{1}{x+1}\) mit \(x+1\) multipliziert. Der Fall \(x = -1\) tritt unter unseren Bedingungen nicht auf}:
\begin{flalign*}
    & \lim_{x \rightarrow +\infty} f(x) = \lim_{x \rightarrow +\infty} \frac{x}{x+1} & \text{| Bruchrechnung} \\
    & \puffer \puffer \puffer \text{ } \text{ } = \lim_{x \rightarrow +\infty} \left(1 - \frac{1}{x+1}\right) & \text{| Kombinationssätze} \\
    & \puffer \puffer \puffer \text{ } \text{ } = \lim_{x \rightarrow +\infty} 1 - \lim_{x \rightarrow +\infty} \frac{1}{x+1} & \text{| Stetigkeit von \(\frac{1}{x}\), auswerten} \\
    & \puffer \puffer \puffer \text{ } \text{ } = 1 - \frac{1}{\lim_{x \rightarrow +\infty} {x+1}} & \text{| Kombinationssatz, auswerten, Schreibweise} \\
    & \puffer \puffer \puffer \text{ } \text{ } = 1 - \frac{1}{{+\infty + 1}} & \text{| Schreibweise} \\
    & \puffer \puffer \puffer \text{ } \text{ } = 1 - \frac{1}{{+\infty}} & \text{| Auswerten} \\
    & \puffer \puffer \puffer \text{ } \text{ } = 1 - 0 \\
    & \puffer \puffer \puffer \text{ } \text{ } = 1
\end{flalign*}
Da \(1 \ne +\infty\) gilt, ist somit gezeigt, was zu zeigen war. \qed \\ \\
\begin{singlespace}
Ich hab noch eine kleine Frage zu \LaTeX: Man sieht im vierten Schritt unserer Auswertung in den Schrittbegründungen, dass es eng wird. Die Fußnote 1 hätte ich auch eigentlich lieber in die Umformung selbst gebracht, aber der Platz geht aus. Die wäre kein Problem, wenn es Zeilenumbrüche gäbe in flalign*, aber es scheint nicht so, da die Elemente sonst "von dem Blatt rutschen". Gibt es hierfür eine elegante Lösung? Ich benutze TeX Live.
\end{singlespace}
\end{document}