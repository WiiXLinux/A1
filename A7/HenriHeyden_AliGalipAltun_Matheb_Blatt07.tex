\documentclass[12pt, a4paper]{article}

\usepackage[ngerman]{babel} 
\usepackage[T1]{fontenc}
\usepackage{amsfonts} 
\usepackage{setspace}
\usepackage{amsmath}
\usepackage{amssymb}
\usepackage{titling}


\newcommand*{\qed}{\null\nobreak\hfill\ensuremath{\square}}
\newcommand*{\puffer}{\text{ }\text{ }\text{ }\text{ }}
\newcommand*{\gedanke}{\textbf{-- }}


\pagestyle{plain}
\allowdisplaybreaks

\setlength{\droptitle}{-11em}
\setlength{\jot}{12pt}

\title{Mathematik für die Informatik B - Hausaufgabenserie 7}
\author{Henri Heyden, Ali Galip Altun \\ \small stu240825, stu242631}
\date{}


\begin{document}
\maketitle

\doublespacing
\section*{Aufgabe 1}
\textbf{Beh.:} Für \(n \in even_1\) ist \(f\) differenzierbar und für \(n \in odd\) ist \(f\) differenzierbar für alle \(x > 0\). \\
\textbf{Bew.:} Für ein beliebiges \(n \in \mathbb{N}_1\) gilt, dass es entweder in \(odd = \{2a+1 | a \in \mathbb{N}\}\) oder in \(even_1 = \{ 2a | 2a \ge 1 \wedge a \in \mathbb{N}\}\) liegt.\\ 
Wir schreiben \(even_1\), da \(n\) aus \(\mathbb{N}_1\) gesucht sind. \\
Hiermit können wir die folgenden Fallunterscheidungen für ein zu überprüfendes \(n\) eröffnen: \\
\textbf{A}: Es gelte \(n \in odd\) und \textbf{B}: Es gelte \(n \in even_1\). \\
Beide Fälle werden wir im Folgenden betrachten:
\subsection*{Erster Fall: A}
Gelte \(n \in odd\) ließe sich \(n\) schreiben als \(2a + 1\) für \(a \in \mathbb{N}\). Dann gelte für \(f_n\) folgendes:
\begin{flalign*}
    & f_n(x) = \sqrt{x}^n & \text{| \(n \in odd\)} \\
    & \puffer \text{ } \text{ } \text{ } = \sqrt{x}^{2a+1} & \text{| Potenzgesetze} \\
    & \puffer \text{ } \text{ } \text{ } = \sqrt{x} \cdot \sqrt{x}^{2a} & \text{| Vereinfache} \\
    & \puffer \text{ } \text{ } \text{ } = \sqrt{x} \cdot x^a \\
\end{flalign*}
Dann gilt: \(f_n(x)' = \frac{1}{2\sqrt{x}} \cdot x^a + \sqrt{x} \cdot ax^{a-1}\) nach der Ableitung der Wurzel (3.45), der Ableitung eines Polynoms (3.42) und der Produktregel (3.41). \\
Bemerke, dass gilt: \(\lim_{x \rightarrow 0} f_n(x)' = +\infty\) gilt (folgt aus den Kombinationssätzen), und somit konvergiert der Differenzenquotient von \(f_n\) nicht an der Stelle 0, womit \(f_n\) an dieser Stelle für \(n \in odd\) nicht differenzierbar ist. \\
\(x = 0\) ist die einzige Stelle bei der \(f_n\) nicht differenzierbar ist, da für \(x > 0\), \(f_n(x)' \in \mathbb{R}\) gilt, was aus den verwendeten Operationen in \(f_n(x)'\) folgt.
\subsection*{Zweiter Fall: B}
Gelte \(n \in even_1\) ließe sich \(n\) schreiben als \(2a\) für \(a \in \mathbb{N}_1\). Dann gelte für \(f_n\) folgendes:
\begin{flalign*}
    & f_n(x) = \sqrt{x}^n & \text{| \(n \in even_1\)} \\
    & \puffer \text{ } \text{ } \text{ } = \sqrt{x}^{2a} & \text{| Vereinfache} \\
    & \puffer \text{ } \text{ } \text{ } = x^a
\end{flalign*}
Nach der Ableitung eines Polynoms (3.42) ist dann \(f_n\) in jeder Stelle differenzierbar mit \(f_n(x)' = a \cdot x^{a-1}\) für \(n \in even_1\). \\ \\
Aus beiden Fallunterscheidungen folgt somit, dass für \(n \in even_1\) ist \(f\) differenzierbar und für \(n \in odd\) ist \(f\) differenzierbar für alle \(x > 0\), \\
\gedanke was zu zeigen war. \qed
\section*{Aufgabe 2}
\textbf{Beh.:} Sei \(\Omega \subseteq \mathbb R\) und \(f : \Omega \rightarrow \mathbb R\) und \(x \in \Omega\) ein HP von \(\Omega\). Dann gilt: \\
\(f\) ist differenzierbar \textbf{(1)} \\
\(\Longleftrightarrow \exists q \in \mathbb{R}^\Omega, q\) ist stetig in \(x: \forall \xi \in \Omega:f(\xi) = f(x) + q(\xi)\cdot(\xi - x)\) \textbf{(2)}\\
\textbf{Bew.:} Um die Äquivalenz beider Aussagen zu zeigen, werden wir den Beweis aufteilen in die Richtungen \textbf{(1) \(\Longrightarrow\) (2)} und \textbf{(2) \(\Longrightarrow\) (1)}.
\subsection*{Richtung (1) \(\Longrightarrow\) (2)}
Sei also angenommen: \(\lim_{\xi \rightarrow x} \frac{f(\xi) - f(x)}{\xi - x} \in \mathbb{R}\).\\
Wir wissen nach Satz 3.36, dass hiermit \(f\) stetig ist. \\
Definiere folgende Funktionen:\\
\(h: \Omega \rightarrow \mathbb R, \xi \mapsto f(\xi) - f(x)\) und \(\tilde{h}: \Omega \rightarrow \mathbb R, \xi \mapsto \xi - x\). Dann sind \(h\) und \(\tilde{h}\) stetig in \(x\) nach Satz 3.20 und der Differenzierbarkeit von Polynomen. \\
Definiere eine weitere Funktion:\\
\(q: \Omega \rightarrow \mathbb R, \xi \mapsto
\begin{small}
    \begin{cases}
        \frac{h(\xi)}{\tilde{h(\xi)}} & \xi \ne x \\
        f'(x) & \xi = x
    \end{cases}    
\end{small}
\) \\
Zuerst zeigen wir, dass \(q\), wenn \(\xi \ne x\) gilt, stetig ist in \(\xi\). \\
Wenn \(\xi \ne x\) gilt, kann \(\frac{h(\xi)}{\tilde{h(\xi)}}\) nicht undefiniert sein, also könnte sie stetig sein. \\
Da \(h\) und \(\tilde{h}\) stetig sind, ist nach Satz 3.20 \(q\) auch stetig in \(\xi\) im Fall \(\xi \ne x\). \\
Dann gilt für alle \(\xi \ne x\) folgendes:
\begin{flalign*}
    & q(\xi) = \frac{h(\xi)}{\tilde{h(\xi)}} & \text{| Einsetzen} \\
    & q(\xi) = \frac{f(\xi) - f(x)}{\xi - x} & \text{| \(\cdot (\xi - x)\)} \\
    & \Longleftrightarrow q(\xi) \cdot (\xi - x) = f(\xi) - f(x) & \text{| \(+ f(x)\)} \\
    & \Longleftrightarrow f(\xi) = f(x) + q(\xi) \cdot (\xi - x)
\end{flalign*}
Somit erfüllt \(q\) für \(\xi \ne x\) die gewünschten Eigenschaften: \textbf{(2)}. \\
Betrachte somit den anderen Fall: Wir zeigen, dass \(q(\xi)\) für \(\xi = x\) stetig ist und dann auch die anderen Eigenschaften von \textbf{(2)} gelten. \\
Wenn \(\xi = x\) gilt, gilt folgendes: \[\lim_{\xi \rightarrow x}q(x) = \lim_{\xi \rightarrow x} \frac{f(\xi) - f(x)}{\xi - x} = f'(x) = q(\xi) = q(x)\]
Dies macht \(q\) stetig in \(x\). \\
Des Weiteren gilt dann auch \[f(\xi) = q(\xi) \cdot (\xi - x) + f(x)\]
da \(f(\xi) = f(x)\) und \(\xi - x = 0\) gelten. \\ \\
Also gilt für \(\xi = x\) die Aussage \textbf{(2)} \\
Somit gilt für alle \(\xi \in \Omega\) die Aussage \textbf{(2)} und diese Richtung ist abgeschlossen. \pagebreak
\subsection*{Richtung (2) \(\Longrightarrow\) (1)}
Sei also angenommen: \(q\) ist stetig in \(x\) und es gilt: \(f(\xi) = f(x) + q(\xi)\cdot(\xi - x)\), \(\xi \ne x\). \footnote{Zum Fall \(\xi = x\) sagen wir gleich noch etwas} \\
Betrachte folgende Umformung:
\begin{flalign*}
    & \puffer \text{ } \text{ } \text{ } f(\xi) = f(x) + q(\xi)\cdot(\xi - x) & \text{| \(-f(x)\)} \\
    & \Longleftrightarrow f(\xi) - f(x) = q(\xi)\cdot(\xi - x) & \text{| \(\cdot \frac{1}{\xi - x}\), angenommen \(x \ne \xi\)} \\
    & \Longleftrightarrow \frac{f(\xi) - f(x)}{\xi - x} = q(\xi) \\
    & \Longrightarrow \lim_{\xi \rightarrow x}\frac{f(\xi) - f(x)}{\xi - x} = \lim_{\xi \rightarrow x}q(\xi) & \text{| q ist stetig in \(x\)} \\
    & \Longleftrightarrow \lim_{\xi \rightarrow x}\frac{f(\xi) - f(x)}{\xi - x} = q(x) \in \mathbb{R}
\end{flalign*}
Also liegt der Differenzenquotient von \(f(x)\) in \(\mathbb R\). Somit ist \(f\) differenzierbar in \(x\). \\
Angenommen \(x = \xi\) gilt, dann können wir beide Seiten nicht durch \(\xi - x\) teilen und dieser Beweis funktioniert nicht. Uns wurde in einer Präsenzübung sogar gesagt, dass das dann überhaupt nicht gilt. Deswegen würde ich argumentieren, dass wir \(\xi\) so beschränken, dass es nicht \(x\) sein darf. Die Begründung aus der Präsenzübung warum \(x = \xi\) nicht gelten darf war, dass dann für ein Folgenglied von einer Folge \((\xi_n)_n\), \(\xi_n = x\) gelte, was jedoch nicht sein darf, wenn wir einen Funktionslimes betrachten wollen, der etwas für die Differenzierbarkeit aussagen soll, wie bei der Definition von Differenzierbarkeit im Skript. \\ Diese Begründung sehen wir ein und sind deswegen davon überzeugt, dass \textbf{(2) \(\Longrightarrow\) (1)}, \textbf{nur} für alle \(\xi \in \Omega \setminus \{x\}\) gelten kann.\footnote{Wir haben in der Behauptung dieses Beweises die Aussage nicht bearbeitet, da wir nicht für Verwirrung sorgen wollten.} \\ \\
Insgesamt sind nun beide Richtungen gezeigt, das heißt es gilt was zu zeigen war. \qed
\section*{Aufgabe 3}
\textbf{Beh.:} Die Aussage aus der Aufgabenstellung gilt nicht, das heißt es existieren \(a,b \in \mathbb{R}\) und \(f: [a,+\infty] \rightarrow [b,+\infty]\) so, dass \(f\) differenzierbar und streng monoton steigend ist und gleichzeitig \(\lim_{x \rightarrow +\infty} f(x) \ne +\infty\) gilt. \\
\textbf{Vor.:} \(a = b = 0\), \(f: [a,+\infty] \rightarrow [b,+\infty], x \mapsto \frac{x}{x+1}\) \\
\textbf{Bew.:} Nach Satz 3.42 können wir eine rationale Funktion wie unser \(f\) ableiten. Hierfür wenden wir ein mal den Quotenentensatz an und bekommen \(f(x)' = \frac{1}{(x+1)^2}\). Dies macht \(f\) differenzierbar in jedem Punkt, denn schließlich gilt für jedes \(x\) aus unserer Domain (\(x \ge 0\)), dass der untere Term der Ableitung nicht 0 werden kann und sonst nimmt die Ableitung auch nur Werte aus \(\mathbb{R}\) an.\\
Nun zeigen wir noch, dass \(f\) auch streng monoton steigend ist.\\
Nach dem Monotoniekriterum in Satz 3.53 zeigen wir, dass für \(f\) im Intervall ihrer Domain ihre Ableitung nur Positive Werte annimmt. \\
Wir zeigen also: \(\forall x > 0: f(x)' > 0\). Betrachte folgende Beobachtung: \\
Gilt \(x > 0\), dann gilt \(x + 1 > 0\). Dann folgt aus den Regeln in angeordneten Körpern, dass \((x + 1)^2 > 0\) gilt und daraus, dass \(\frac{1}{(x+1)^2} > 0\) gilt. \\
Somit gilt: \(\forall x > 0: f(x)' > 0\), also ist \(f\) streng monoton steigend. \\
Nun haben wir alle Voraussetzungen (der Aufgabenstellung) erfüllt, also werden wir zeigen, dass \(\lim_{x \rightarrow +\infty} f(x) \ne +\infty\) für unser \(f\) gilt. \\
Betrachte somit folgende Auflösung des Limetes\footnote{Die Gleichheit beim ersten Schritt sieht man leicht, indem man beide Seiten von \(\frac{x}{x+1} = 1 - \frac{1}{x+1}\) mit \(x+1\) multipliziert. Der Fall \(x = -1\) tritt unter unseren Bedingungen nicht auf}:
\begin{flalign*}
    & \lim_{x \rightarrow +\infty} f(x) = \lim_{x \rightarrow +\infty} \frac{x}{x+1} & \text{| Bruchrechnung} \\
    & \puffer \puffer \puffer \text{ } \text{ } = \lim_{x \rightarrow +\infty} \left(1 - \frac{1}{x+1}\right) & \text{| Kombinationssätze} \\
    & \puffer \puffer \puffer \text{ } \text{ } = \lim_{x \rightarrow +\infty} 1 - \lim_{x \rightarrow +\infty} \frac{1}{x+1} & \text{| Stetigkeit von \(\frac{1}{x+1}\) für \(x \ge 0\)} \\
    & \puffer \puffer \puffer \text{ } \text{ } = 1 - \frac{1}{\lim_{x \rightarrow +\infty} (x+1)} & \text{| Kombinationssatz, auswerten, Schreibweise} \\
    & \puffer \puffer \puffer \text{ } \text{ } = 1 - \frac{1}{{+\infty + 1}} & \text{| Schreibweise} \\
    & \puffer \puffer \puffer \text{ } \text{ } = 1 - \frac{1}{{+\infty}} & \text{| Auswerten nach Schreibweise} \\
    & \puffer \puffer \puffer \text{ } \text{ } = 1 - 0 \\
    & \puffer \puffer \puffer \text{ } \text{ } = 1
\end{flalign*}
Da \(1 \ne +\infty\) gilt, ist somit gezeigt, was zu zeigen war. \qed \pagebreak
\begin{singlespace}
Ich hab noch eine kleine Frage zu \LaTeX: Man sieht im vierten Schritt unserer Auswertung in den Schrittbegründungen, dass es eng wird. Die Fußnote 1 hätte ich auch eigentlich lieber in die Umformung selbst gebracht, aber der Platz geht aus. Die wäre kein Problem, wenn es Zeilenumbrüche gäbe in flalign*, aber es scheint nicht so, da die Elemente sonst "von dem Blatt rutschen". Gibt es hierfür eine elegante Lösung? Ich benutze TeX Live.
\end{singlespace}
\end{document}