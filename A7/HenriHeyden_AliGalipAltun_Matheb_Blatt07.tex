\documentclass[12pt, a4paper]{article}

\usepackage[ngerman]{babel} 
\usepackage[T1]{fontenc}
\usepackage{amsfonts} 
\usepackage{setspace}
\usepackage{amsmath}
\usepackage{amssymb}
\usepackage{titling}


\newcommand*{\qed}{\null\nobreak\hfill\ensuremath{\square}}
\newcommand*{\puffer}{\text{ }\text{ }\text{ }\text{ }}
\newcommand*{\gedanke}{\textbf{-- }}


\pagestyle{plain}
\allowdisplaybreaks

\setlength{\droptitle}{-11em}
\setlength{\jot}{12pt}

\title{Mathematik für die Informatik B - Hausaufgabenserie 7}
\author{Henri Heyden, Ali Galip Altun \\ \small stu240825, stu242631}
\date{}


\begin{document}
\maketitle

\doublespacing
\section*{Aufgabe 1}
\textbf{Beh.:} Für \(n \in even_1\) ist \(f\) differenzierbar und für \(n \in odd\) ist \(f\) differenzierbar für alle \(x > 0\). \\
\textbf{Bew.:} Für ein beliebiges \(n \in \mathbb{N}_1\) gilt, dass es entweder in \(odd = \{2a+1 | a \in \mathbb{N}\}\) oder in \(even_1 = \{ 2a | 2a \ge 1 \wedge a \in \mathbb{N}\}\) liegt.\\ 
Wir schreiben \(even_1\), da \(n\) aus \(\mathbb{N}_1\) gesucht sind. \\
Hiermit können wir die folgenden Fallunterscheidungen für ein zu überprüfendes \(n\) eröffnen: \\
\textbf{A}: Es gelte \(n \in odd\) und \textbf{B}: Es gelte \(n \in even_1\). \\
Beide Fälle werden wir im Folgenden betrachten:
\subsection*{Erster Fall: A}
Gelte \(n \in odd\) ließe sich \(n\) schreiben als \(2a + 1\) für \(a \in \mathbb{N}\). Dann gelte für \(f_n\) folgendes:
\begin{flalign*}
    & f_n(x) = \sqrt{x}^n & \text{| \(n \in odd\)} \\
    & \puffer \text{ } \text{ } \text{ } = \sqrt{x}^{2a+1} & \text{| Potenzgesetze} \\
    & \puffer \text{ } \text{ } \text{ } = \sqrt{x} \cdot \sqrt{x}^{2a} & \text{| Vereinfache} \\
    & \puffer \text{ } \text{ } \text{ } = \sqrt{x} \cdot x^a \\
\end{flalign*}
Dann gilt: \(f_n(x)' = \frac{1}{2\sqrt{x}} \cdot x^a + \sqrt{x} \cdot ax^{a-1}\) nach der Ableitung der Wurzel (3.45), der Ableitung eines Polynoms (3.42) und der Produktregel (3.41). \\
Bemerke, dass gilt: \(\lim_{x \rightarrow 0} f_n(x)' = +\infty\) gilt, und somit konvergiert der Differenzenquotient von \(f_n\) nicht an der Stelle 0, womit \(f_n\) an dieser Stelle für \(n \in odd\) nicht differenzierbar ist. \\
\(x = 0\) ist die einzige Stelle bei der \(f_n\) nicht differenzierbar ist, da für \(x > 0\), \(f_n(x)' \in \mathbb{R}\) gilt, was aus den verwendeten Operationen in \(f_n(x)'\) folgt.
\subsection*{Zweiter Fall: B}
Gelte \(n \in even_1\) ließe sich \(n\) schreiben als \(2a\) für \(a \in \mathbb{N}_1\). Dann gelte für \(f_n\) folgendes:
\begin{flalign*}
    & f_n(x) = \sqrt{x}^n & \text{| \(n \in even_1\)} \\
    & \puffer \text{ } \text{ } \text{ } = \sqrt{x}^{2a} & \text{| Vereinfache} \\
    & \puffer \text{ } \text{ } \text{ } = x^a
\end{flalign*}
Nach der Ableitung eines Polynoms (3.42) ist dann \(f_n\) in jeder Stelle differenzierbar mit \(f_n(x)' = a \cdot x^{a-1}\) für \(n \in even_1\). \\ \\
Aus beiden Fallunterscheidungen folgt somit, dass für \(n \in even_1\) ist \(f\) differenzierbar und für \(n \in odd\) ist \(f\) differenzierbar für alle \(x > 0\), \\
\gedanke was zu zeigen war. \qed
\end{document}